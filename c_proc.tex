%
% the file c_proc.tex
%


The Laboratory assigns an end station (Hall~A, Hall~B, or Hall~C) to 
each experiment
upon granting approval. The hall's scientific and technical groups are the
experiment's primary
contacts with the Laboratory. The hall staff provides experiment support,
such as beamlines,
detector components, engineering and technical expertise, and office space.
The host Hall Leader
and the experiment spokesperson are responsible for all environment, safety
and health
(EH\&S) aspects of the experiment.

Hall~C Procedures for Experiments serves three purposes:
\begin{enumerate}
\item{It is the primary policy document that describes how experiments are
managed by JLab Hall~C throughout their life cycle.}

\item{It describes the Hall~C and experiment collaboration responsibilities
for conducting JLab physics experiments in a safe and environmentally sound 
manner.}

\item{It provides a guide to experimenters to ensure that EH\&S
considerations are incorporated early in the design stage of the 
experiment.}
\end{enumerate}

\section{Safety Responsibilities of Experimenters}


The experiment Spokespersons are responsible for ensuring that all 
members of the collaboration which work in the experimental area have read 
the relevant {\em Conduct of Operations (COO)} and 
{\em Experimental Safety Assessment Document (ESAD)}.  Spokespersons shall 
retain a record of this ``training" and 
supply it to the Associate Director for Physics, the Physics Division EH\&S Officer, and
the Hall~C Leader upon request.

\subsection{Obtaining Required Safety Training}

Before performing any work at the laboratory, experimenters must 
receive a basic safety
orientation from the Users Office. Besides this general safety training,
all personnel must
receive job specific training. For instance, all persons must receive General Employee
Radiological Rraining (GERT) before unescorted access to the accelerator site,
and must have radiation Worker I training and a JLab radiation dosimeter before 
working in Radiation Controlled Areas or working with radioactive materials or sources.
Level II Radiation Worker Training is required for working with radioactive 
contamination.

Hall~C is a Radiation Controlled Area and contains radioactive material, hence all 
experiment personnel must have a JLab dosimeter or badge when entering a Radiation 
Controlled Area or handling radioactive materials.  
To work in Hall~C you must complete three ``core" EH\&S courses, view
the Hall~C
Access Video, and take an escorted Hazard Awareness Walk-Through of the
hall. The core
courses are EH\&S Orientation to JLab, Oxygen Deficiency Hazard
Training, and Radiation
Worker Training. You must complete each of these courses before receiving a
JLab
identification badge which allows unescorted access to the accelerator
site. Without taking these
courses, you may not enter the accelerator site without an authorized
escort, and you may not
work in the hall itself.

EH\&S Orientation to JLab provides an overview of JLab's
environmental health
and safety philosophy, and an introduction to important EH\&S concerns
including, but not
limited to: hazard identification and communications, risk assessment,
emergency management,
stop work procedures, and concern resolution. The course also defines the
EH\&S
responsibilities of individuals working at JLab.

Oxygen Deficiency Hazards (ODH) Training provides training to 
individuals who work
in areas such as Hall~C where cryogens and gases have a significant
potential for creating an
oxygen deficiency hazard.

Radiation Worker Training provides an overview of radiation protection
principles as
they relate to JLab radiation safety procedures. It discusses principles
of radiation protection,
radiation absorption, biological effects, survey instrumentation, the
laboratory's radiation control
program, and emergency procedures. Once you have completed this course, you 
may apply for
a thermolumiscent dosimeter or TLD; you must have a TLD in order to enter
the hall
unescorted.

EH\&S course training can be received through one of two methods:
attending lecture
classes and taking the subsequent examinations, or taking the proficiency
challenge exams
administered by the User Liaison Office and the Radiation Control Group.
The proficiency
exams are open to anyone who wants to challenge the standard course
lectures. Users who opt
for the proficiency exam approach but fail the exam will be required to
take the standard lecture
presentations and any related required tests and practicums.

Proficiency exam study materials are currently available through the 
User Liaison Office
and online via the World Wide Web. The User Liaison Office also plans to
have each course on
video. For additional information, you may contact the User Liaison Office
by phone at
extension 7586 or in person in JLab Center, Room A204.

Finally, an escorted Hazard Awareness Walk-Through of Hall~C will review
access
procedures, identify hazardous equipment, and review general safety
precautions for working in
the hall. The three basic courses and walk-through will be supplemented
with Apparatus-
Specific Training as required for specific responsibilities you may have
for your experiment.
Examples of work requiring Apparatus-Specific Training are work on
cryogenic targets and
polarized targets.

This list is not all-inclusive and many experiments have additional
special training
requirements, for example, the operation of Polarized or Cryogenic target
Systems. All
experiment personnel are required to have radiation badges in their
possession during shifts and
when in Hall~C. All entries into the Hall during an experiment require the
two man rule be
obeyed. In addition, all shift personnel must be trained on the safety
procedures to be followed
for access to the hall and its close-up prior to beam delivery.

\subsection{The Personnel Safety System}

Users and staff working on the accelerator site are protected from the
dangers associated
with the prompt ionizing radiation that the accelerator beam produces by
JLab's Personnel
Safety System or PSS. The PSS keeps ionizing radiation out of areas where
people are
working, and keeps people out of areas where ionizing radiation is present.
For example, when
the PSS is in the Restricted Access state, people are allowed to work in
the hall and measures
are in place to prevent delivery of beam to the hall. If the PSS is in the
Beam Permit state, beam
can be delivered to the hall and measures are in place to prevent people
from entering the hall.

The Personnel Safety System includes Run/Safe boxes which are located
throughout Hall
C, and approximately every 100 feet in the linac. A run/safe box has three
positions: Safe,
Operational, and Unsafe. When the hall is in Restricted Access, the
run/safe box will be in the
Safe position. While in this position, the PSS prevents delivery of beam to
the hall. Before
beam can be delivered, the hall must be swept to ensure that no one is left
inside. During the
Sweep, each run/safe box is moved to the Operational position in
preparation for Beam Permit. After the sweep has been completed and the
hall is placed in the
RF Power Permit state, the run/safe box will show Unsafe. Each box has an
emergency stop
button. If you see the box in the Unsafe position, you are in danger of
receiving high levels of
ionizing radiation. Immediately press the emergency stop button, exit the
hall, and call the
Machine Control Center Crew Chief at extension 7050.

There are a total of five states for the Hall~C Personnel Safety System:
Restricted Access,
Sweep, Controlled Access, RF Power Permit, and Beam Permit.

\paragraph{Restricted Access} is when delivery of beam and/or RF power is not
permitted, and entry
to and exit from the hall is not controlled by the Personnel Safety System.
This is the normal
state of the hall when the accelerator is off and no experiments are running.
Access is
``restricted" only in the sense that the hall is not open to the general public.

\paragraph{Sweep} is when delivery of beam and/or RF power is not permitted and 
access is limited
to the JLab personnel conducting the sweep operation. The hall's entrance
gates are closed
from the inside to ensure that no one can enter behind the person
conducting the sweep. During
the sweep, an Assigned Radiation Monitor or ARM systematically searches the
hall to verify
the absence of people and to arm the run/safe boxes. The ARM posts a guard
at the entrance to
the hall as another method of ensuring that no one enters after him.


When the Assigned Radiation Monitor is ready to perform a sweep, the
Machine Control
Center or MCC must first place the hall in the Sweep state. The Personnel
Safety System will
read ``Sweep In Progress." Once the hall is placed in the sweep state, the
sweep monitors enter
the first gate to the hall, making sure it locks behind them. The ARM then
notifies the MCC that
he is ready to begin the sweep. The MCC communicates with the sweep
monitors via intercom
and video camera. Using the video camera, the MCC makes sure both sweep
monitors are
wearing the proper dosimetry. At this point the ARM also indicates that he
is in possession of
the key needed to arm the Run/Safe boxes placed throughout the hall.

Having confirmed that the dosimetry is adequate, the MCC will unlock the
second
entrance gate allowing the sweep monitors to enter the hall. Once the sweep
monitors pass
through the second gate, they close the gate and ensure it is locked. The
sweep monitors then
proceed to the hall entrance where one sweep monitor is left to guard the
entrance and the other
begins the sweep.

During the actual sweep, the ARM walks through every area and secluded
workspace in
the hall to ensure that no one could be left inside when the Personnel
Safety System moves
from the sweep state to controlled access, power permit, and finally beam
permit state. Once he
checks an area, he arms the run/safe box in that area.

After all areas of the hall have been checked and the run/safe boxes
armed, the sweep
monitors will return to the entrance where the sweep began. Before arming
the last run/safe box,
the ARM will contact the MCC. Upon contact, the MCC will check to see if
the sweep has
``dropped"; if all is well he will notify the ARM that it is okay to arm the
box. Once the box is
armed, the sweep monitors have 30 seconds to exit both gates or the sweep
will drop, and the
entire sweep process will have to be repeated. After exiting, the ARM must
contact the MCC to
let them know the Hall~Can now be moved to the controlled access state.

\paragraph{Controlled Access} is when delivery of beam and/or RF power is not
permitted but the hall
is considered a controlled area.  In this state, people are ``counted" both
entering and leaving the
hall to ensure that no one is left inside when the Personnel Safety System
advances to the RF
Permit or Beam Permit states.  Hall entry during the controlled access
state is permitted only to
people authorized or qualified by JLab.  Entry to and exit from the hall
is controlled from the
MCC.  The Hall cannot be placed in the ``controlled access" state without
having first been
swept.

\paragraph{RF Power Permit} is when the hall is considered an ``exclusion area."
Delivery of RF
power is permitted,  but beam delivery is not.  Reaching this state
requires that the hall has
passed through the controlled access state and that no one is left inside
the hall. This is usually a
temporary state bridging the transition from the Controlled Access to the
Beam Permit state. Once
the Personnel Safety System reads ``Power Permit," a steady klaxon sounds in
the hall. If you
are in the hall when this klaxon sounds, press the emergency safe button on
the nearest run/safe
box and immediately exit the hall. The hall entrance gates are locked at
this time, but there is an
emergency exit button at each gate which will allow you to exit. A
four-minute delay is built in
between the transition from RF Power Permit to Beam Permit.

\paragraph{Beam Permit} is when delivery of beam and RF power is permitted to the
exclusion area.
Reaching this state requires having passed through the RF Power Permit state.
\subsection{Hall~C Access}

Access to Hall~C will be governed as described in the ``JLab Beam 
Containment
Policy and Implementation" document dated March 1994, with the modification
that controlled
access entrances and exits will be monitored directly by an ARM (assigned
radiation monitor)
provided through the JLab Accelerator Division or a JLab Radiation
Control (RadCon)
Group member, and not solely via a television camera. Work in designated
radiation areas will
be governed by the JLab RadCon Manual.

Access procedures during Research Operations depend on the number of
individuals who
will be entering the hall and the length of time they are expected to be there.

A controlled access is used when a few individuals require entry for a
short period of
time. If the hall must be open for an extended period and many people will
enter, then you
should use the restricted access procedure instead of the controlled access
procedure.  Normally,
when requesting a controlled access, the hall will be in either the Beam
Permit or RF Permit
State - for example, if the beam has been on or it could be shortly.

If the hall is not already in the Controlled Access state when you 
wish to access it, you
must request a change to that state from the Machine Control Center at
extension 7050 and
indicate that you intend to make a Controlled Access. The MCC will then
send an Assigned
Radiation Monitor to survey the hall. Before anyone enters the hall, the
ARM will carry out a
radiation survey and post radiation areas. Subsequent entry by individuals
during the same
Controlled Access period does not require an ARM survey.

\subsubsection{Controlled Access Procedure}
To make a controlled access when the hall is in the controlled access
state, first contact the MCC. The MCC will unlock the first gate at
the entrance to the hall. Once inside, the MCC will release the master
key. Remove the master key and insert it into the right-most slot of
the row of keys below it. Once the master key is in place, each person
wishing to gain access must remove a key from this row. The MCC will
then verify each person's name, which key he has, and check that each
person is wearing the proper dosimetry. This key-release procedure
allows the MCC to keep a ``count" of who has entered the hall. After
the  procedure is complete, the MCC will unlock the second gate at the
entrance to the hall. Please note: only one of the entrance gates can
be open at a time while in the controlled access state.

When your work is completed and you are ready to exit, return to the
entrance gates and press the intercom call switch to notify the
MCC. Once you have entered and closed the first gate, each person must
replace his key in the appropriate slot, otherwise the Personnel
Safety System will not allow the master key to be released. When the
master key is released, place it in its slot, and the MCC will unlock
the final gate. When you have exited the final gate, make sure it has
closed and locked behind you. If circumstances dictate, request that
the MCC return the hall to the beam permit state and that beam be
restored.

It is important to note that if you enter the HMS shield house during
the controlled access, the run/safe box inside the shield house will
drop from the operational state to the safe state as soon as the door
to the shield house opens. Unless this box is rearmed by an MCC
operator, the beam cannot run.  Further, before the door can be
opened, a protective cover moves into place over the HMS spectrometer
vacuum window.  Therefore, it is necessary to call MCC and request
that they come to the hall and resecure the HMS shield house when you
are finished with your access.  The operator will enter the shield
house, rearm the security system, close the shield house door, and
withdraw the protective shutter. Make certain that the operator
performs all of these steps before she/he leaves the hall.

\subsubsection{Restricted Access Procedure}
Restricted Access is used when the hall will be open for an extended
period of time or
a large group will enter to work. To drop the hall to the
Restricted Access state, first
notify the MCC that you wish to open the hall in the Restricted Access
state. The MCC will
drop the hall status to Controlled Access and send an ARM to survey the
hall. Before anyone
can enter the hall, the ARM will carry out a radiation survey and post
radiation areas.  The hall is
placed in Controlled Access during the survey to ensure that no one enters
before it has been
completed. Upon completion of the survey and posting of radiation areas,
the ARM will leave
the hall and notify the MCC that they can drop the hall state to Restricted
Access. With the hall
in the Restricted Access state, anyone with the appropriate training may
enter and work.  The
key- release procedure is not required.


To return the hall to Beam Permit from the Restricted Access state, a 
full inspection must
be carried out. This is begun by setting all equipment to its operating
state (following the Hall~C
checklist) and then clearing all workers out of the hall. Next, a request
is made to the MCC to
arrange a sweep of the hall and to restore the Beam Permit state. The MCC
will send over an
ARM and set the hall status to Sweep.  The ARM will then sweep the hall,
verifying that
everyone is out. Following a successful sweep, the MCC can move the hall
through the
Controlled Access and RF Permit states to the Beam Permit state.

While working in the hall you must observe all posted radiation areas.
Remember, work
inside a radiation area requires that you obtain an approved radiation
permit. You must also
observe the ``two-man" rule, and pay attention to the alarms.

\subsubsection{Badge Reader Physical Access Control}

General physical access to Hall C is restricted by a full time badge reader system. The badge
reader limits non-emergency hall access to those individuals on the approved access lists. The
Hall Leader and experiment Liaison Physicist maintain the data base, with input from the
experiment run coordinator, the Hall C work coordinator, Hall C safety warden and physics division
safety personnel. As a part of the general access control the Liaison Physicist working with the collaboration
management will collect names of those who state by signature that they have read and understood
the COO and ESAD.

The badge reader based security system is in addition to the engineering and administrative
controls discussed previously. Specifically, to gain physical access to the hall
requires the logical .AND. of all engineering based access control systems. If the hall is in a 
"Restricted Access" or lesser state the maglock will release after a valid badge is scanned by
the badge reader. Each individual must scan his/her badge separately and all
entries and exits are logged. Badges and access privileges are assigned to individuals. Letting
individual(s) into the hall via a badge not assigned to them will be treated as the circumvention
of a laboratory safety system. Exceptions include formal prearranged and approved guided tours or
escorting of a visitor who has a RADCON issued dosimeter. 

If the hall is in "Controlled Access" those seeking entry must also
request access with MCC (generally by the phone near the door) simultaneously with the badge
scanning to unlock the outer Hall C personnel door. The MCC cannot override the badge reader's
data base and a valid badge does not guarantee that the MCC will allow entry into Hall C. The
badge reader's data base of authorized individuals is not static and may be modified as
appropriate for the activities underway in the hall at that time.


\subsection{Attire}

During periods when heavy construction is in progress, all workers 
are required to wear long pants, shirt sleeves,
eye protection, a hard hat, and steel-toed shoes - sandals are not allowed. 
Signs at the entrance to the
hall will identify whether it is under heavy construction.

When you access the hall for brief periods of time in Controlled Access
during an
experimental run or for work in the hall that is not during periods of
heavy construction, you are
still required to wear a hard hat. Other personal protective equipment
must be worn as
required for the job you intend to do - for example eye protection, gloves,
or other protective
equipment should be worn as appropriate for tasks like electrical work or
welding.  A hard hat
may be temporarily removed in situations where it creates larger hazards
than it mitigates - for
example, when working around the wire chambers in the detector hut.
However, the hard hat
must be worn as you travel through the hall to the hut. The wearing of long
pants and steel-toed
shoes is recommended.

\subsection{Hall~C Specific Hazards}

There are several Hall~C-specific hazards which you will see in your
required walk-
through. These include: the SOS Power Supply, the HMS Power Supply, the
magnetic field in
the first quadrupole and the first dipole of the HMS and SOS when the power
is on (flashing
red lights indicate when power is on); the target area; the SOS large
vacuum window; the Hall~C
cyrogenic target system; the beam dump; the steps going up to the beam line;
the HMS detector hut
vacuum window; the drift chamber windows; possibly damaged equipment; high 
power-low voltage or
high-voltage electronic equipment; the high-voltage feed to PMTs; and slip,
trip, and fall
hazards. Additional hazards may be present for specific experiments. The
specific experiment's
safety assessment document must contain these details.

Detailed rules specifying who is authorized to work on what equipment
within the hall
are listed in the \htmladdnormallink{ESAD}{http://www.jlab.org/Hall-C/document}." 

\subsection{Alarms}

The alarms in Hall~C include: radiation alarms, ODH alarms, and fire 
alarms.


\subsubsection{Radiation Alarms}
Radiation Alarms are located outside the experimental halls, for example
in the access
labyrinth. A magenta beacon indicates the presence of a possible radiation
hazard. If you hear an
audible alarm, a radiation  hazard has been detected. Exit the area as soon
as possible. Once you
are safely out of the area, contact the Machine Control Center, call 911 if
emergency assistance
is required, and call extension 4444 to notify JLab emergency personnel.
Always keep
personal safety your first priority.

\subsubsection{ODH Alarms}
ODH Alarms are both in the hall in general (for cryogenic ODH) and in 
the HMS shield
house (for chamber gas). A blue beacon indicates the presence of a possible
ODH hazard -
proceed with caution. If an ODH hazard is detected, an audible alarm will
sound. Exit the area
as soon as possible. Do not attempt to rescue co-workers inside unless you
have the proper
equipment and training. Once you are safely out of the area, contact the
Machine Control Center,
call 911 if emergency assistance is required, and call extension 4444 to
notify JLab
emergency personnel. Always keep personal safety your first priority.

\subsubsection{Fire Alarms}
 In case of fire, immediately exit the area. As you exit, 
pull the nearest fire
alarm. Once you are safely out of the area, call 911 for emergency
assistance, and call extension
4444 to notify JLab emergency personnel. If the fire is small enough to be
controlled, you
may use a fire extinguisher. Do this only if you have been properly trained
and elect to do so.
Always keep personal safety your first priority.

JLab is firmly committed to protecting the health and safety of its
staff, users, and
contractors. Ensuring that people are aware of hazards - and education in
the proper operating
and access procedures - is an important part of meeting this commitment.
Your awareness of
the hazards in the workplace and the care with which you carry out your
work are the first line
of defense against accidents and injuries. Always remember that NO activity
is so urgent that
safety may be compromised.

\subsection{General Issues}


\paragraph{Know and follow safety rules} and other requirements in the JLab
EH\&S
Manual, the JLab Radiological Control Manual and the Hall~C Equipment
Operating
Manuals. These manuals are available in the Physics Division. Hall~C
Operating Manuals may
also be viewed on the World Wide Web. Also see the ``Pearls of Wisdom" in
Appendix~\ref{app:pearls}.


\paragraph{Report unsafe working conditions} to the spokesperson, a Hall~C safety
warden,
Physics Division EH\&S personnel, or any JLab organization such as the
Physics Division.


\paragraph{Reporting of incidents.}   The laboratory is required to notify the DOE of
significant
incidents such as activation of an automatic sprinkler system, loss of
radioactive material, failure
of safety equipment, violation of safety procedures, significant
environmental releases such as
spills of hazardous substances, high radiation exposures or contamination,
and substance abuse.
Experimenters should notify any Physics Division personnel of such incidents.


\paragraph{Following tour requirements}.  Normally only registered experimenters,
authorized
contractors or subcontractors and JLab employees may enter experimental
areas. Physics
Division EH\&S personnel should be contacted to obtain the current the
policy for conducting
tours in these areas.

\section{Physics Division in the Life of an Experiment}


Physics Division supports experiments from ``cradle to publication."
Section 3120 of the EH \&S Manual, and, on the web (www.jlab.org), the Procedures
for Nuclear Physics Experiments, describe Physics Division 
responsibilities and services to an
experiment during its life.

\section{Control of Equipment and System Status}


During the running of Hall~C experiments, significant
modifications, repairs, and/or changes to the operating status of equipment
may only be done by the
resident ``experts" or persons under their direct supervision. Note:
significant changes to the
operating status of equipment means changes beyond the nominal operating
parameters of the
subsystems. These ``experts" are primarily the JLab staff members or users
who were
principals in designing, building and testing the subsystems. More details,
the current lists and
procedures for adding to the list of experts individuals are available in
the ESAD. 
This document expanded and
tailored to the
specific experiment should accompany the user generated ``Conduct of
Operations Document
(COO)." ``Operational Safety Procedures for Hall~C"and the ``Safety
Assessment Document
for Hall~C Base Equipment"  are the core operations documents for Hall~C equipment.

\subsection{Monitoring and Emergency Shutdown of Hall~C Systems}


The HMS superconducting magnets require special monitoring and control
procedures.
The ``Operational Safety Procedures for Hall~C" containing these procedures
are located in the
Hall~C Counting House on the console next to the magnet control
screens/keyboards.   The instructions will enable the shift personnel
to perform routine
operation of the magnets during Hall~C commissioning. In addition, there
is a set of seven
large labeled red emergency shut down ``crash" buttons and two key switches
in the center of
the Hall~C control room console.


The buttons have the following functions:

\begin{enumerate}

\item Emergency Accelerator Shutdown (activated when beam is enabled for 
Hall~C).

\item Emergency power supply shutdown for superconducting quadrupole Q1.

\item Emergency power supply shutdown for superconducting quadrupole Q2.

\item Emergency power supply shutdown for superconducting quadrupole Q3.

\item Emergency power supply shutdown for HMS superconducting dipole.

\item Emergency shutdown of all short-orbit-spectrometer (SOS) magnet power
supplies.

\item {Emergency shut down of all ``clean" power in the end station - kills
clean power in the HMS shield house where the detector components are 
located.}
\end{enumerate}

The keys have the following functions:

\begin{enumerate}

\item{Disables power to all four HMS superconducting magnet power 
supplies.  Essentially, this key electrically locks the four HMS crash buttons 
to their in (crash) positions. It 
does not  disable power to the cryogenic systems or magnet control 
computer.}

\item{Disables power to all three SOS magnet power supplies and the moller
magnet power supply. Essentially, this key electrically locks the SOS crash 
buttons to their in (crash) positions.}

\end{enumerate}


There are also TV cameras (displayed at the Hall~C console) viewing the
HMS detector, the general Hall area, the Hall~C personnel access door, the 
targets, and the beamline phosphor viewers.



\subsection{Independent Verification - Hall~C Checklists}


We require that one of the following basic check lists be
completed and placed in the Log Book prior to closing Hall~C for beam.

\newpage
\subsubsection{Typical Check List - Prior to Hall~C Production Running}

\noindent Counting Room Status (Hall~C counting room must be manned.):
\vspace{\baselineskip}

\noindent End station Status:
\vspace{\baselineskip}

\noindent Items to check prior to securing the Hall for beam:

\begin{itemize}
\item[{[~~~~]}] Check status of Drift chamber gases - in gas shed.
\item[{[~~~~]}] Confirm that drift chamber DC. power in the shield house is on.
\item[{[~~~~]}] Confirm that Hodoscope, drift chamber, and Pb-glass shower counter 
H.V.
is on.
\item[{[~~~~]}] Confirm that all crates in shield houses are on and enabled.
\item[{[~~~~]}] Confirm that runs can be started/stopped.
\item[{[~~~~]}] Confirm that spectrometer(s) are set at proper angles and leveled.
\item[{[~~~~]}] Confirm that the correct selection of targets is in scattering chamber.
\item[{[~~~~]}] Confirm that target motion system is operational.
\item[{[~~~~]}] Confirm beamline to dump is configured appropriately for 
energy/current
being used.
\item[{[~~~~]}] Confirm clean/safe status of shield house area and Hall.
\item[{[~~~~]}] Confirm ``OK" for current on status of superconducting magnets - 
console.
\item[{[~~~~]}] Set currents in magnets for first run, and confirm.

\item[{[~~~~]}] Confirm that any hall ventilation which exhausts air to the environment is turned off.
\item[{[~~~~]}] Confirm that beamline and target chamber are free from unnecessary equipment and clutter.
\item[{[~~~~]}] Confirm that beam line and target structures have been wiped down in the last six months or since the last maintenance activity to remove dust and grease which presents a source of loose radioactivity.

\item[{[~~~~]}] Close up the Hall.
\end{itemize}
\vspace{0.5in}
\hspace*{\fill}{\underline{~~~~~~~~~~~~~~~~~~~~~~~~~~~~~~~~~}}
\newline
\hspace*{\fill} {Signature~~~~~~~~~~~~~~~~~Date} 

\newpage
\subsubsection{Typical Check List - When Hall~C is in a Standby Mode}

\noindent Counting Room Status (Manning of Hall~C counting room is optional):
\vspace{\baselineskip}

\noindent Prior to closing the Hall~Check the End station Status:
\vspace{\baselineskip}

\begin{itemize}

\item[{[~~~~]}] Check status of Drift chamber gases - in gas shed.
\item[{[~~~~]}] Confirm that drift chamber DC. power in the shield house is on.
\item[{[~~~~]}] Confirm that Hodoscope, drift chamber, and Pb. glass shower counter 
H.V. is on.
\item[{[~~~~]}] Confirm that all crates in shield houses are on and enabled.
\item[{[~~~~]}] Confirm that runs can be started/stopped.
\item[{[~~~~]}] Confirm that spectrometer(s) are set at proper angles and leveled.
\item[{[~~~~]}] Confirm that correct targets are in scattering chamber.
\item[{[~~~~]}] Confirm that target motion system is operational.
\item[{[~~~~]}] Confirm beamline to dump is configured appropriately for 
energy/current
being used.
\item[{[~~~~]}] Confirm clean/safe status of shield house area and Hall.
\item[{[~~~~]}] Confirm that all magnet power supplies are shut-off and locked/tagged at breakers in 
Hall~C or alternately the Hall~C console key for the HMS power supplies are in 
the ``off" position with key removed. The key must be removed from the counting 
house and held in the possession of one of the following individuals: R. 
Carlini, W. Vulcan, P. Brindza, S. Lassiter, or M. Fowler.

\end{itemize}

\vspace{0.5in}
\hspace*{\fill}{\underline{~~~~~~~~~~~~~~~~~~~~~~~~~~~~~~~~~}}
\newline\hspace*{\fill} {Signature~~~~~~~~~~~~~~~~~Date}


Monitoring of superconducting magnets cryogenics done according to 
normal standby mode procedures.

\newpage
\subsubsection{Typical Check List -  When Hall~C is Used as Accelerator Beam Dump}

\noindent Counting Room Status (Hall~C counting room may be unmanned.):
\vspace{\baselineskip}

\noindent Prior to closing the Hall~Check End station Status:
\vspace{\baselineskip}

\begin{itemize}
\item[{[~~~~]}]
Clean power in Hall~C is to be turned off.
\item[{[~~~~]}]
Spectrometers rotated out of beam direction.
\item[{[~~~~]}]
Targets in extracted position.
\item[{[~~~~]}]
Drift Chamber gases are turned off at gas shed.
\item[{[~~~~]}]
Confirm clean/safe status of shield house area and Hall.
\item[{[~~~~]}]
Beamline to dump configured appropriate for beam energy/current 
\item[{[~~~~]}] 
Confirm that any hall ventilation which exhausts air to the environment is turned off.
\item[{[~~~~]}]
 Confirm that beamline and target chamber are from unnecessary equipment and clutter.
\item[{[~~~~]}] 
Confirm that beam line and target structures have been wiped down in the last six months or since the last maintenance activity to remove dust and grease which presents a source of loose radioactivity.
being used.
\item[{[~~~~]}]  Confirm that all magnet power supplies are shut-off and locked/tagged at breakers in 
Hall~C or alternately the Hall~C console key for the HMS power supplies are in 
the ``off" position with key removed. The key must be removed from the counting 
house and held in the possession of one of the following individuals: R. 
Carlini, W. Vulcan, P. Brindza, S. Lassiter, or M. Fowler.
\end{itemize}

\vspace{0.5in}
\hspace*{\fill}{\underline{~~~~~~~~~~~~~~~~~~~~~~~~~~~~~~~~~}}
\newline\hspace*{\fill} {Signature~~~~~~~~~~~~~~~~~Date}


Monitoring of superconducting magnets cryogenics done according to 
normal standby mode procedures.

