%
% the file criteria.tex
%
\noindent Devices or systems which meet the following criteria must undergo an EH\& S review 
before initial operation.

\paragraph{Mechanical Hazards:} Devices which meet any of the following criteria:

\begin{enumerate}
\item Weighs over 3 tons and is supported above the floor

\item Exceeds 10 tons in total weight

\item Moves at a speed greater than 5 ft/sec

\item Costs more than \$100,000 to replace

\item Includes pressure/vacuum vessels
\end{enumerate}

\paragraph {Flammable Gas Systems:}  

Any use of flammable gas and flammable gas mixtures.

\paragraph {Electrical Hazards:}  

Electrical systems which meet any of the following criteria:

\begin{enumerate}

\item Uses non-commercial or modified commercial equipment.

\item Uses non-PREP or modified PREP equipment.

\item Any non-commercial low voltage high current or high voltage distribution
systems.

\item Any equipment with large capacitor banks.

\end{enumerate}

\paragraph {Fire Hazards:}  Any large combustible items such as large quantities of 
plastic scintillator, large numbers of cables requiring cable trays

\paragraph {Oxygen Deficiency Hazards:} Use of any oxygen displacing gases such as 
chamber gas systems, helium bag systems, dry nitrogen, cryogenic magnets or targets

\paragraph {Cryogenic Hazards:} Cryogenic systems for magnets, hydrogen targets, 
calorimeters, or any cryogenic system with inventory exceeding 200 liters.

\paragraph {Laser Hazards:} Lasers of any class.

\paragraph {Radiation Hazards:} Radioactive sources/materials which will be used.
Specify if embedded in detectors.

\paragraph {Toxic Materials:}  Toxic/hazardous materials planned or used, if the
amount exceeds few gallon/pound quantities. Examples include: lithium, beryllium, 
mercury, lead, uranium, and cyanide.

