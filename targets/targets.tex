\chapter{Targets}

\section{The Hall~C Scattering Chamber}

The scattering chamber consists of a large central band made from a single
forged ring of 6061-T6 Al. This ring has an inner diameter of
48.5 inches and a  2.5 inch wall (0D = 53.5 inch). The ring has
cutouts consistent with the vertical angular acceptances and the
full angular ranges of the two Hall~C spectrometers. In addition,
the cutout for the SOS has been made to accommodate 20 degrees
of out of plane movement. There are also openings through which the beam
will enter and exit, a pumping port, several ports for viewing, and
some ports for as yet unspecified purposes.

There are two shells which are attached to the outside of the central
band. These are cylindrical sections which have an inner diameter
equal to the outer diameter of the scattering chamber. These shells
are designed to carry sliding vacuum seals so that the scattering
chamber vacuum can be directly coupled to that of the two spectrometers.
The SOS shell is mounted on a set of ``sliders" and can be moved
up and down by a series of linear actuators in order to accommodate the
out of plane motion of the spectrometer.

In the early stages of operation these shells are replaced by
clamps which hold thin fixed metal vacuum windows.

The opening for the SOS is five inches tall. The metal window is
5052-H39 aluminum 0.008 inches thick. This is the same foil that was used to
cover the five inch tall window on the temporary Hall~C scattering chamber
(The SLAC chamber). This is the same material that was used by SLAC when
the chamber was in use there.
The opening for the HMS
is eight inches tall. This will be spanned by a window of
5052-H34 aluminum 0.016 inches thick. The tank has been pumped down with
{\bf both openings covered by 0.008 inch thick 5052-H39 }
and vacuum cycled several
times. The crinkling pattern was examined and the inter crinkle spacing
was used as input for stress calculations. In this approximation the
total window is treated as a collection of smaller windows each of which
has a height equal to that of the full opening and a width given by the average
inter crinkle spacing. These calculations indicate that the windows
proposed above (0.008 inch for SOS and 0.016 inch for HMS) have at least
a factor of two safety margin.

These windows have been tested to failure with over pressure from the
outside.  The average failure pressure exceeded 40 psid and thus they
have a safety margin of over 2.5.  

There are also top and bottom plates which complete the main body of the
scattering chamber.

The bottom plate allows the chamber to be mounted to the
solid shaft which forms the pivot axis for the two spectrometers.


The top plate has a number of openings. The largest of these allows
the cryotarget plumbing and lifting mechanisms into the vacuum and is sealed by
a large diameter bellows. There is also a three inch diameter tube through
which the solid targets are inserted.
The other three openings are capped by eight inch diameter aluminum
conflats and currently have no specified function.

The beam entry and exit tubes are vacuum coupled to the scattering chamber with
metal seals. The beam exit tube terminates in a 0.015 inch thick beryllium foil.
This window should be regularly inspected for signs of deterioration.

%All questions about the scattering chamber or its vacuum windows should
%be referred to J. Mitchell -x7851.

\section{The Solid Target Ladder}

The scattering chamber has two target ladders associated with it,
a cryogenic target ladder and a solid target ladder. To change
from the cryogenic target ladder to the solid target ladder, one has
to:

\begin{itemize}
\item{Move the cryogenic target ladder to its home position.}
\item{Rotate the cryogenic target ladder over 90 degrees. This removes
the cryotargets from the beam line, and moves them out of the spectrometer
acceptances.}
\item{Insert the solid target ladder and select the appropriate target.}
\end{itemize}

\noindent To install the cryogenic target ladder again, one has to reverse the
process:

\begin{itemize}
\item{Move the solid target ladder to its home position (completely out
of the way).}
\item{Rotate the cryogenic target ladder back into place (-90 degree
rotation).}
\item{Select the appropriate target.}
\end{itemize}

We have four different solid target systems currently in use in the
Hall C Scattering Chamber.  There is a solid target attached directly
to the cryotarget and three versions of the independent solid target
ladder system.

The solid target that is attached to the bottom of the cryotarget is
often referred to as the optic target.  This target is a system of
aluminum and carbon foils arranged at different locations and heights
relative to the beam.  The foils provide cryotarget window
calibration and loop target spectrometer optics calibration.  The
various targets are brought into the beam by simply closing the
vertical height of the cryotarget stand and a ``scan'' can be made
that spans the length of the cryotarget in discrete steps on window
convection data taken.  This system replaced the earlier ``slanted
target'' system.

The independent solid target system has three different ladders which
mount interchangeably to the same mechanism.  This mechanism
provides for a telescopic guidance system with good accuracy
and stability as well as vertical, and rotary control.  The rotary mechanism also
has an integral rotary water feed thru for high power solid target
runnings. 

The three solid target ladders use identically sized interchangeable
variable thickness solid targets.  They use a common target mounting
system.  The three ladders currently available are:

\begin{itemize}
\item{Un-cooled target ladder with a standard solid target position.
This is for 0-30 $\mu$A rastered beams for materials with good
conductivity and high melting points.}
\item{Water cooled target ladder with 6 standard solid target
positions.  This is for 0-100 $\mu$A beams that are rastered.  These
can handle any material.}
\item{Combination water cooled solid and water target.  This target
has, in addition to 5 solid target positions, a new water cell.
The water cell is 1 cm thick and has entrance and exit windows of 0.1
mm (~.005 in) aluminum.  The water cell is all machined and welded to
the bottom of the solid target ladder.}
\end{itemize}

In some cases the solid targets will be cooled with water to prevent
destruction of the targets by a high current beam. If this is the case
make sure that the
cooling water is indeed flowing before the experiment starts.
{\sl Failure to do so may cause loss of the targets}. Often we
do not have to use water cooling. This is the case when we are
rastering the beam, or are dealing with solid target
materials with excellent thermal conductivities (like pyrollytic
graphite or ceramic BeO) and currents less than or equal to 30 $\mu$A.

In all cases, as we may use these targets in conjunction with a high
energy, high current beam a large level of radiation may be induced
in these targets.

The main safety issue concerning the non-special targets is that
they will be radioactive (``hot") after they have been
exposed to the JLab beam for an experiment.
Targets may not be removed from the target chamber ot transported from the Hall without 
concurrence by the RCG. An SRWP specifies the requirements for target removal and/or 
transport. Targets which have been surveyed and released by the RCG will be stored in a 
second fire safety cabinet which has been designated for target storage and is located in 
the hall.

Since the targets may be water cooled, we will induce a slight
radioactive level in the cooling water. However, the cooling water
system is a closed loop, and calculations performed by the Radiation
Safety group indicate that the radioactive level induced is minimal.
Also, to ensure that the cooling water is indeed a contained system
and can not leak out to the environment, the pumping system causing
the water to flow is placed in a tray.

\subsection{Special Targets}

Some of the solid targets require special attention. In some cases
mishandling of the targets is relatively innocent, like in the case
of the natural {\bf Calcium} target. This material, when exposed to air,
will oxidize, resulting in a large oxygen content in the target.
Store natural Calcium always in an argon-filled or vacuum closed
case, or in an oil solution. Calcium can be handled in air for a limited
amount of time (a few hours) if handled with a layer of oil.

Other special targets may pose a safety concern. At this
moment the only twos special target materials we own are ceramic
{\bf Beryllium-Oxide (BeO)} and {\bf Beryllium (Be)}.
In solid form, BeO is completely safe under normal conditions of use.
The product can be safely handled with bare hands.
However, in powder form all Beryllia is {\bf toxic} when airborne.
{\bf Overexposure to airborne Beryllium particulate may cause a
serious lung disease called Chronic Berylliosis. Beryllium has also
been listed as a potential cancer hazard. Furthermore exposure to
Beryllium may aggravate medical conditions related to airway systems
(such as asthma, chronic bronchitis, etc.).}
Since beryllia are mainly dangerous in powdered form, do not machine,
break, or scratch these products. Machining of the Beryllia can only
be performed after consulting the EH\&S staff.
It is good practice to wash your
hands after handling the ceramic BeO. If handling the pure Beryllium
target wear gloves and an air filter mask.

\subsection{Storage}

The targets {\bf must} be stored in a safe, locked place tagged with the
appropriate radiation signs. At this moment we use a small yellow
fire safety cabinet for this purpose. The cabinet is stored in the
little backroom of the Hall~C counting house. The key of the target
storage cabinet can be found in the Hall~C key box, hanging on the
wall next to the Hall~C counting house entrance.
Targets may not be removed from the target chamber or transported from the hall without 
concurrence by the RCG.  An SRWP in the counting house specifies the requirements for 
target removal and/or transport.  Targets which have been surveyed and released by the 
RCG will be stored in a second fire safety cabinet placed downstairs in Hall~C,
that has been designated for target storage.

Since we will typically have
Be and/or BeO targets stored in this cabinet too, also signs indicating
the presence of Beryllium materials (``Beryllium Storage")
are tagged to this cabinet. The Material
Safety Data Sheets for Beryllia are included in the storage cabinet.


\section{The Cryogenic Target System}

Much of the physics program at JLab requires the use of cryogenic targets
filled with hydrogen, deuterium or helium isotopes. The Hall~C cryogenic
system contains three separate loops in order to allow for rapid changes
of target fluid.  This system will be extensively discussed in
another section.  Here, we only summarize some main features.

A target loop is a circulating system of target gas and consists of the following
elements:

\subsection{Target Circulation Fan} This is some sort of device to keep
the target fluid circulating through the system. In the Hall~C cryogenic
targets this function is performed by a two stage axial flow fan which is
immersed in the target fluid. This is a small AC motor with fan blades attached
on both shafts of the rotor. The blades in our case are simple Archimedes
screws with a diameter of 2.8 inches. The motors where extracted from Globe
VAX-3-FC Blowers, part $\#$ 19A798. The factory installed
bearings were then replaced with bearings suitable for cryogenic service
(Barden, Bartemp-no lube SR4SSTB5). The fans are powered asynchronously
with three phase AC power from a variac and since the stators are wound with
two poles the maximum rotation speed is 3600 rpm.
\subsection{Target Cells} This is the thin walled vessel in the circulation loop
where the electron beam interacts with the target fluid. Two types
of target cells will be employed in Hall~C. These are the ``beer can" cells
and the vertical flow cells. The vertical flow cells are also referred
to as ``tuna can" or ``sink trap" cells.
The beer can cells are constructed from Coors beer cans (3004 series aluminum)
and come in two standard lengths, 4 cm and 15 cm.
The vertical flow cells are 4 cm in length and are constructed from 7050
series aluminum.
\subsection{Heat Exchangers} This is where the heat that is deposited in the target
fluid by interaction with the electron beam is removed. The target fluid is
circulated on one side of the heat exchanger and cold helium
from a refrigerator is circulated through the other side of the heat exchanger.
The cryogens for the Hall~C target will be supplied from the End Station
Refrigerator (ESR).

A new feature in 1999 is the conversion of the cryotarget cooling from
the previous series/parallel flow to a pure parallel flow cooling
system.  This was done to make control of the three cryotarget loops
simpler, truly independent, and truly interchangeable.  The use of
LN$\_2$ as a precooler was elected in favor of directing a small
amount of the 20K exhaust helium.  

\subsection{Heaters} The temperature of the target can be regulated by powering
a heater immersed in the target fluid.

\subsection{Other Features of the Cryotarget System}  In addition to the above components which make up the loop a cryogenic target
needs a gas handling system and instrumentation. The gas handling system
enables the operator to fill, empty and perform other manipulations of the
target while the instrumentation is needed in order to verify the target's
status, temperature, pressure, etc. The temperature is particularly critical
in that it is the dominate parameter that is correlated to target density.

A system of several targets also needs a motion mechanism so that the desired
target can be inserted in the beam. The Hall~C target accomplishes this by
means of a three rail system, two ball screws and a guide bearing.
The ball screws are driven by two AC motors each of which has a resolver and
a 50 to 1 gear reducer. The resolver output of one of the motors is used
to track the position with one of the motor controllers acting as a master
while the second controller is slaved to the master.

All the instrumentation and control of the target system is implemented
via the EPICS system (this is the same control system used by the main
machine).

%Questions concerning the cryogenic targets should be referred to J.H. Mitchell
%at extension 7851.

%\vfil
%\eject

\section{Bremsstrahlung Radiator}

The Bremsstrahlung radiator consists of a target ladder with
several thicknesses of natural Cu of size 1.5" times 0.75".
Typical radiator thicknesses are 2\%, 4\%, and 6\%, all in
radiation lengths. The Bremsstrahlung radiator is the last element
in the Hall~C beamline before the scattering chamber, at a distance
of about 1.2 meter from the physics targets.

The target ladder can be moved in and out of the beam by use of
a stepping motor. Targets are cooled with water to prevent destruction
of the targets by a high current beam.
Because we may use these radiator targets in conjunction with a high
energy, high current beam we may produce a large amount of background
in the radiator, and radiation in the parts around the radiator.
For this reason the Bremsstrahlung radiator will be enclosed by a lead
shield.

The only safety issue concerning the Bremsstrahlung radiator is that of
induced radioactivity in
the Cu targets, or more seriously, in the water used for cooling the targets.
 The water cooling system is a closed loop. The heart of the system is
a portable welding torch water cooler. This device is kept in a tray
which is intended to provide secondary containment in case of a leak.
 The cooling system must not be breached or drained without concurrence from the RCG.  
Accidental breach or spill constitutes a radiation contamination hazard.  A spill control kit, 
capable of containing a system leak or spill, is staged in the hall next to the cooling system 
tray.  The cooling system is located    .

The Cu targets will certainly be activated in the course of an experiment.
Therefore, only remove
the Cu target, the target ladder, the shielding around the radiator,
and/or the whole radiator system in the presence of a Radcon technician.

\paragraph{Radiator Control}

In one of the xterm windows, type the following:

\begin{verbatim}
	cd ~cdaq/bin
	Radcontrol
\end{verbatim}

and use the menu session to control the motion of the
Bremsstrahlung radiator.

The source code can be found in:

\begin{verbatim}
	cd ~cdaq/bin/Rad_source
\end{verbatim}

Do not change the source code without consulting
%Dave Meekins or
one of the JLab Hall~C staff.
