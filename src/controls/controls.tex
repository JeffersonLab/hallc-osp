A distributed computer system
based on the 
Experimental Physics and Industrial Control System 
(EPICS)~\cite{EPICSwww}
%\htmladdnormallinkfoot{}{\url{http://www.aps.anl.gov/epics}}
 architecture monitors and commands
the various Hall C systems. The basic components of the system are:
\begin{itemize}
\item Input/Output Controllers (IOCs) - Systems containing single
board computers (SBCs) and I/O modules
(i.e analog-to-digital converters (ADCs), digital I/O and RS-232C interfaces).
Each SBC executes the real-time operating system VxWorks and the corresponding EPICS application (signal database
and sequencers).  IOCs are typically VME SBCs and crates, but may also
be other platforms such as standard Linux servers or small form factor
systems such as Raspberry PIs.
\item Operator Interfaces (OPI) - Computers capable of executing
EPICS tools to interact with the IOCs.
The four most used tools in Hall C are (a)
a Web-enabled version of the Motif-based Display Editor/Manager (MEDM)~\cite{MEDMwww}, 
(b) StripTool and, (c) a signal archiver.
MEDM is the main interface used for monitoring and controlling both the hall and accelerator
equipment. StripTool allows to monitor 
the behavior of one or more signals as a function of time. 
The signal archiver keeps a record of a selected set of signals.
\item Boot Servers - IOCs load the various
software components needed to perform their functions from these machines (i.e. operating system,
signal database and controls algorithms).
%\item MEDM Servers - OPI computers obtain the framework of each MEDM screen from these machines.
%\item Local Area Network (LAN) - the communication path joining the IOCs, OPIs and various servers.
\end{itemize}

In addition equipment controlled with EPICS, Hall C uses an industrial
control system, composed of Allen Bradley PLCs (Programmable Logic
Controllers) and commercial OPI software.  This system is used to
control the cryogenics, magnets and motion of the HMS and SHMS
spectrometers.  The use of this system is described in
chapter~\ref{spectrometerschapter}. 

\infolevtwo{
\section{System's Components}

Need to be careful not to put information here that quickly goes out
of date.  Refer to howtos for EPICS build directory, HV configuraion
and operation

Basic description of EPICS.

Note that used to display information about accelerator status and to
display and control Hall C equipment such as targets and high voltage.

Describe logging of EPICS information.   Information logged by
accelerator.  Instructions on how to add signals to logging.  How to
display logged information.

Note that EPICS information is recorded in CODA data stream.  

Account used for EPICS.  Location of control screens.

Refer to target chapter and target operating manual for target controls.

Note that spectrometer magnet and angle setting does not use EPICS.
Refer to spectrometer chapter for information on that control system.
Note that spectrometer information is availabe as readonly EPICS so
that spectrometer information can be read, displayed and logged with
EPICS tools.

}
