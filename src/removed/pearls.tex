%
% the file pearls.tex
%

Following is a list of do's and don't's for experimenters working in Physics division. We 
have gleaned these pearls from experience.

\subsubsection*{The Basics:}
	
\begin{itemize}
\item All experiment personnel are required to have radiation badges in their possession when 
inside the accelerator fence as per JLab's beam containment policy. All entries into the 
Hall during an experiment require the two man rule be obeyed. In addition, all shift 
personnel must be trained on the safety procedures to be followed for access to the hall 
and its close-up prior to beam delivery.

\item Do not bring any radioactive sources onto the site without first getting permission from 
the EH\&S Section.

\item If it becomes necessary to perform work on any piece of equipment in the hall that may 
have received a significant dose of direct radiation, {\bf you must have it surveyed by a 
radiation monitor before your work can begin}.  This means that all work on beamline or 
target chamber equipment will involve a radiation survey.  

\item If it is necessary to remove any item from the hall after or during a running period, it 
should be surveyed first.  {\bf All components must be surveyed and released by a 
radiation officer before they can be removed from the hall.} 

\item Be sure to lock and tag equipment before maintaining it.

\item Do not use or turn on any equipment which meets the hazard criteria in Appendix III, until it has been reviewed. Contact your Spokesperson or Liaison Physicist for
 assistance.

\item Do not bring any chemicals on site without a Material Safety Data Sheet (MSDS).  
Ensure that the safety warden gets a copy of the MSDS for all new chemicals.

\item Use care in proper disposal of unused or waste chemicals. Don't throw them in the 
drain!  
Contact the safety warden or the Physics Division Environmental Officer if you are 
uncertain how to dispose of them.

\item As a general rule, do not work in technical areas alone. If this is not possible, contact 
Operations first.

\item Call Operations before making any controlled access. 

\item Physics division encourages the use of equipment from completed experiments, however 
this equipment is not necessarily ``free for the taking." Contact Physics division personnel 
such as the liaison physicist, safety warden or call the Physics Division Office before 
taking any equipment.

\item Do not use machine shop tools or other power equipment unless you are properly trained 
or qualified.

\item JLab scientists and Users are not permitted to operate fork lifts or cranes under any 
circumstances.
\end{itemize}

\subsubsection*{Electrical:}

\begin{itemize}

\item All electrical wiring must be sized and fused appropriately for the current it will carry to 
prevent electrical fires. ``Home built" low-voltage high-current power supplies for wire-
chamber readout electronics typically have a single ground return. The ground return 
must be sized to carry the current of all supplies that use the common return.

\item Do not lay signal cables in power cable trays. Power cable trays contain magnet or other 
high current power cables. Signal cables must be either laid in a separate (signal cable) 
tray or must be separated from the power cables by an approved fire barrier. Consult your 
safety warden before laying cables.

\item Do not lay or attach flammable gas lines in any cable tray (power or signal).

\item Do not lay or attach extension cords or SO type cord in cable trays.

\item Do not modify or add to a building electrical distribution system. All electrical work must 
conform to the National Electrical Code and be performed by qualified electricians.

\item Do not use cube taps to expand the number of available 110 VAC outlets.  Plug strips 
with circuit breakers are available in the stock-room and are allowed.
\end{itemize}

\subsubsection*{Mechanical:}

\begin{itemize}

\item All mechanical components and devices such as support stands, frames, transporters or 
targets built by groups outside JLab must be coordinated with the Mechanical Support 
Department and undergo reviews by JLab engineers.

\item All flammable gas and cryogenic systems must be designed and/or reviewed by the 
Mechanical Support Department. A review must be performed before fabrication and/or 
installation.

\end{itemize}
