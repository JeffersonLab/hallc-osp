%
% the file a_quest.tex
%

	This appendix provides guidance in defining experiment administration during 
data taking. JLab experimental groups are made up of collaborating institutions, each 
of which typically has a senior member serving as group leader, as well as some number 
of post-docs and students. Often the chain of authority within the collaboration is not 
traditional; it is often organized around the components of the detector, rather than around 
the senior physicists. The formality and structure required within a group depends on the 
size and complexity of the collaboration. JLab experiments range in size from small 
activities with fewer than ten participants to international collaborations of more than 
100.  Traditionally, the spokesperson is responsible for reporting the scientific status of 
the experiment to JLab management and the physics community at large. Today this 
responsibility is often delegated to other members of the collaboration. However, the 
spokesperson of each experiment is ultimately accountable to the director for the safe 
operation of his/her experiment.

	We provide the following set of questions for evaluating your experiment's 
administration. Senior members of the experiment should determine whether to codify the 
answers in a document for use by shift personnel. These questions are posed within the 
context of DOE Order 5480.19, "Conduct of Operations Requirements for DOE 
Facilities".

	There are no right or wrong answers to these questions. The experiment 
collaboration should determine the documentation needs for shift personnel based on the 
answers to these questions and the complexity of the detector.

\section{General}

\begin{itemize}
\item Is there a description of the responsibilities of each collaborating
institution (i.e. Memorandum of Understanding)?

\item   What is the method for communicating the experiment goals and standards
to all members of the collaboration?

\item   What is the method for communicating EH\&S policies (such as this document) 
and procedures to all members?

\item   What is the method for delegating responsibilities from the spokesperson to other 
members?

\item   What is the method for delegating responsibilities from system experts to other 
members?

\end{itemize}

\section{Shift Routines and Operating Practices}

\begin{itemize}
\item   Is experiment specific hazard awareness training given to all personnel working on 
the detector?

\item   Is basic experiment operation training given to on-shift personnel (i.e. starting and 
stopping runs.)?

\item   Is there a document that describes the shift structure?

\item   Is there a description of the responsibilities of shift personnel (i.e. shift checklists 
and walk abouts)?

\item   Is there a procedure describing how to respond to (and reset) safety alarms?

\item   Is there a procedure describing how to respond to (and reset) equipment failure 
alarms?

\item   Is there a procedure describing how to respond to (and reset) data quality alarms?

\end{itemize}

\section{Control Room Activities}


\begin{itemize}
\item   What is the policy regarding the occupancy of the control room?

\item   What is the policy regarding the behavior of off-shift personnel in the control 
room?

\end{itemize}

\section{Communications}


\begin{itemize}
\item   Is there a need for broadcasting emergency information in the experimental area?

\item   Will emergency information be understood by collaborators who do not speak 
English?


\end{itemize}

\section{Control of On-Shift Training}

\begin{itemize}
\item   Is the operation of the experiment complex enough to require a formal training 
program for on-shift personnel?

\end{itemize}

\section{Investigation of Abnormal Events}


\begin{itemize}
\item   Are on-shift personnel properly trained to respond to and report safety related 
occurrences?

\item   Are on-shift personnel properly trained to respond to equipment failures?

\item   Are on-shift personnel properly trained to troubleshoot problems?

\item   Are there any diagnostic procedures for troubleshooting problems?

\end{itemize}

\section{Notifications}


\begin{itemize}
\item   Is it clear to on-shift personnel whom to notify when a safety concern exists?

\item   Is it clear to on-shift personnel when a system expert should be called?

\item   Should the spokesperson be called when the experiment is down for an extended 
period?

\end{itemize}

\section{Control of Equipment and System Status}


\begin{itemize}
\item   Are there procedures for changing the operating parameters of the detector and 
beamline (i.e. changing triggers, targets and beam conditions)?

\item   Are all operating limits clearly specified for all equipment (i.e. chamber voltages, 
pressures, temperatures, flow rates)?

\item   Are there methods for checking these limits?

\item   Is there a procedure for tagging defective equipment and ensuring that it is 
returned for repair?

\item   Is there a status board that displays the current state of the experiment?

\end{itemize}

\section{Lockouts and Tagouts}


\begin{itemize}
\item   Do procedures exist for any equipment that needs to be  locked and/or tagged for 
repair or maintenance?

\end{itemize}

\section{Independent Verification}


\begin{itemize}
\item   Is there a mechanism for verifying the integrity of the data (i.e. off-line analysis)?

\end{itemize}

\section{Logkeeping}

\begin{itemize}
\item   What is the standard for making logbook entries (i.e. legibility, use of ink, signed 
and dated entries, photocopy reproducible, correction methods)?

\end{itemize}

\section{Operations Turnover}


\begin{itemize}
\item   Is there a need for a formal shift changeover (i.e. summary of past shift status and 
current problems.)?

\item   Are shift summaries recorded in the logbook?

\end{itemize}

\section{Interactions with Support Personnel}


\begin{itemize}
\item   Does a call list exist for on-shift personnel to request information or assistance on 
1) beam - line/accelerator performance, 2) environmental controls (HVAC), 3) technician 
services, 4) rigging services?

\end{itemize}

\section{Required Reading}


\begin{itemize}
\item   What is the mechanism for ensuring that on-shift personnel are kept current on 
new or changed procedures?

\end{itemize}

\section{Operating Procedures}


\begin{itemize}
\item   Are all operating procedures located in a single binder or location?

\item   Is there a procedures index?

\end{itemize}

\section{Equipment Labeling}

\begin{itemize}
\item   Are the following components clearly labeled: modules, detector elements, power 
supplies, cables, gauges?

\end{itemize}




