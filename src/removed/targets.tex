\section{The Cryogenic Target System}

Much of the physics program at JLab requires the use of cryogenic targets
filled with hydrogen, deuterium or helium isotopes. The Hall~C cryogenic
system contains three separate loops in order to allow for rapid changes
of target fluid.  This system will be extensively discussed in
another section.  Here, we only summarize some main features.

A target loop is a circulating system of target gas and consists of the following
elements:

\subsection{Target Circulation Fan} This is some sort of device to keep
the target fluid circulating through the system. In the Hall~C cryogenic
targets this function is performed by a two stage axial flow fan which is
immersed in the target fluid. This is a small AC motor with fan blades attached
on both shafts of the rotor. The blades in our case are simple Archimedes
screws with a diameter of 2.8 inches. The motors where extracted from Globe
VAX-3-FC Blowers, part $\#$ 19A798. The factory installed
bearings were then replaced with bearings suitable for cryogenic service
(Barden, Bartemp-no lube SR4SSTB5). The fans are powered asynchronously
with three phase AC power from a variac and since the stators are wound with
two poles the maximum rotation speed is 3600 rpm.
\subsection{Target Cells} This is the thin walled vessel in the circulation loop
where the electron beam interacts with the target fluid. Two types
of target cells will be employed in Hall~C. These are the ``beer can" cells
and the vertical flow cells. The vertical flow cells are also referred
to as ``tuna can" or ``sink trap" cells.
The beer can cells are constructed from Coors beer cans (3004 series aluminum)
and come in two standard lengths, 4 cm and 15 cm.
The vertical flow cells are 4 cm in length and are constructed from 7050
series aluminum.
\subsection{Heat Exchangers} This is where the heat that is deposited in the target
fluid by interaction with the electron beam is removed. The target fluid is
circulated on one side of the heat exchanger and cold helium
from a refrigerator is circulated through the other side of the heat exchanger.
The cryogens for the Hall~C target will be supplied from the End Station
Refrigerator (ESR).

A new feature in 1999 is the conversion of the cryotarget cooling from
the previous series/parallel flow to a pure parallel flow cooling
system.  This was done to make control of the three cryotarget loops
simpler, truly independent, and truly interchangeable.  The use of
LN$\_2$ as a precooler was elected in favor of directing a small
amount of the 20K exhaust helium.  

\subsection{Heaters} The temperature of the target can be regulated by powering
a heater immersed in the target fluid.

\subsection{Other Features of the Cryotarget System}  In addition to the above components which make up the loop a cryogenic target
needs a gas handling system and instrumentation. The gas handling system
enables the operator to fill, empty and perform other manipulations of the
target while the instrumentation is needed in order to verify the target's
status, temperature, pressure, etc. The temperature is particularly critical
in that it is the dominate parameter that is correlated to target density.

A system of several targets also needs a motion mechanism so that the desired
target can be inserted in the beam. The Hall~C target accomplishes this by
means of a three rail system, two ball screws and a guide bearing.
The ball screws are driven by two AC motors each of which has a resolver and
a 50 to 1 gear reducer. The resolver output of one of the motors is used
to track the position with one of the motor controllers acting as a master
while the second controller is slaved to the master.

All the instrumentation and control of the target system is implemented
via the EPICS system (this is the same control system used by the main
machine).

%Questions concerning the cryogenic targets should be referred to J.H. Mitchell
%at extension 7851.

%\vfil
%\eject
