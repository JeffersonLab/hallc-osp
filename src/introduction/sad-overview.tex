\infolevone{
\chapter[Hall A Hazards and Safety Assessment Overview]{Hall C Safety Assessment Overview~\footnote{
Authors: D. Higinbotham {doug@jlab.org} and originally E.Chudakov}
}
}
 
\begin{safetyen}{0}{0}
\section{Overview of the Hazards} 
\end{safetyen}
\label{sec:overviewhazards}

This section is gives a general overview of the hazards one may encounter
while in Hall C,
without going into the details of each part of the equipment.  In order to 
be able to operate a particular piece of equipment in a safe way one must study the appropriate
section of  
\infolevltthree{the full OSP manual\cite{HallCosp}.}
\infolevthree{this OSP manual.}
The general hazards are:
\begin{list}{\arabic{enumi}.~}{\usecounter{enumi}\setlength{\itemsep}{-0.15cm}}
  \item Radiation hazard (see Sec.\ref{sec:radhazard});
  \item Fire hazard (see Sec.\ref{sec:firehazard});
  \item Electrical hazard (see Sec.\ref{sec:electrhazard});
  \item Mechanical hazard (see Sec.\ref{sec:mechhazard});
  \item Hazard from strong magnetic fields (see Sec.\ref{sec:maghazard});
  \item Cryogenic and Oxygen Deficiency Hazard (ODH) (see Sec.\ref{sec:odhhazard});
  \item Vacuum and high pressure hazards  (see Sec.\ref{sec:vachazard});
  \item Toxic materials hazard  (see Sec.\ref{sec:toxichazard}).
\end{list}

The principal contacts for Hall C safety issues are given in Tab.\ref{tab:ehs:principlecont}.
 
\begin{namestab}{tab:ehs:principlecont}{Principle contacts for safety issues}{%
   Principle contacts for safety issues}
  \WalterKellner{\em Safety Warden}
  \BertManzlak{\em EHS Engineer}
\end{namestab}
 
\begin{safetyen}{0}{0}
\section{Radiation Hazard} 
\label{sec:radhazard}
\end{safetyen}
  The radiation hazards and the ways to mitigate them are described in detail in the
  course of Radiation Worker I (RW-I) training~\cite{RWIcebaf},
  as well as in the Hall A Radiation Work Permit (RWP).
  Here, the most essential issues are discussed.

  CEBAF's high intensity, high energy electron beam 
  is a potentially lethal radiation source and hence many redundant measures,
  called Personnel Safety System or PSS~\cite{PSScebaf}, are in place,
   aimed at preventing accidental exposure of personnel to the beam or exposure 
  to beam-associated radiation sources.
  The PSS keeps ionizing radiation out of areas where people are working,
  and keeps people out of areas where ionizing radiation is present. 
  The PSS procedure to enter the hall is described in detail in  Sec.~\ref{sec:Access}.

%  Here, we remind of a few items.

% \begin{list}{--}{\setlength{\itemsep}{-0.cm}}
%    \item 
 All of Hall C is qualified as a ``Radiologically Controlled Area''~\cite{RWIcebaf}.
 Entrance requirements are listed in Sec.\ref{sec:access-req}.
 Some areas, such as the target area, and the area around the beam dump may be qualified as 
 a ``Radiation Area'' or a ``High Radiation Area''. Some areas may be also qualified 
 as a ``Contaminated Area', if removable radio-isotopes are likely to be present.
 These areas should be marked with appropriate signs and may be delimited by barrier. 
 Access to ``Radiation Areas'' requires a permit from RadCon, while access
 to ``High Radiation Area'', and ``Contaminated Area'', is not allowed. One should consult 
 the RWP document for more details.

 All the items, except those kept in the shielded detector huts of HRS,
          which stayed in the Hall during CW beam operations,
          must be surveyed and released by a qualified Radiological Control Technologist from
          RadCon group, prior to removal from the hall.
          A rack close to the entrance is used to store these items.

 Some electronic modules and racks are posted as potentially contaminated.
 They must be surveyed and, if necessary, cleaned by RadCon personnel,
 prior to removal from the Hall, or prior to performing any work on the internal parts of the racks
 and modules, including the air filters. 
 
 More details on radiation safety issues can be found in various sections of this 
 document, as in Chap.\ref{sec:targets-overv} for target operations and
 in Sec.\ref{sec:beam-intro} for operation of the equipment on the beam line.
 
 The contacts for Hall C radiation safety issues are given in Tab.\ref{tab:ehs:radiation}.
 
 \begin{namestab}{tab:ehs:radiation}{Contacts for radiation safety issues}{%
   Contacts for radiation safety issues}
  \WalterKellner{\em Safety Warden}
  \RadCon{\em RC Group}
 \end{namestab}


\begin{safetyen}{0}{0}
\section{Fire Hazard} 
\label{sec:firehazard}
\end{safetyen}

 Fire Hazards are associated with the use of electrical power and also with the use of 
 flammable gases and/or materials. 

 The flammable gasses include the cryotarget
 materials such as hydrogen or deuterium 
 (see the details on the hazard and its mitigation in target section of full OSP \infolevone{Sec.\ref{sec:target-cryo-safety}}) as well as
 the gas used in the wire chamber detectors of the spectrometers 
 (see the details on the hazard and its mitigation in full OSP \infolevone{Sec.\ref{sec:hrs-det-gasalarms}}).
 
 In general an effort has been made to limit the volume of combustible material 
 in the hall but some flammable material is unavoidable. For instance all plastic 
 scintillators are flammable and if exposed to a direct flame these 
 plastic materials will eventually melt. The elements then lose structural integrity, 
 sag or fall to the floor, and the melted elements would likely be exposed to air and burn.

% Some special equipment in the subsystems, like heaters, lasers etc., may present a
% fire hazard (see Sec.\ref{sec:targ-polhelfire}\infolevone{ and Sec.\ref{sec:heaterinterlock}}).
 
\textbf{
UPDATE for HALL C
The fire hazard in Hall C is mitigated by a VESDA smoke detection system. 
% The head sensitivity of 
% the VESDA is 0.03 to 0.003\%. This corresponds to approximately 1 % of full scale and the system trips 
% at 90\% full scale. 
 The main VESDA panel is located in the room at the bottom of the truck ramp on the 
 right hand side as you walk out of Hall A. The clean power in the detector huts is 
 interlocked to the VESDA system. If the VESDA system senses smoke, it will remove power 
 from the huts.
}

 The detector huts are equipped with a clean agent fire suppression system. This system, when triggered 
 by a smoke detector installed on the hut ceiling, releases an inert gas mixture into the hut and dilutes 
 the oxygen level below that needed to sustain combustion. The inert gas is a mixture of nitrogen, 
 argon and carbon dioxide. When the system is functioning properly the oxygen content of the air in the 
 hut will be reduced to approximately 12.5\% from the standard 21\%. Operation of this system
 would result in an ODH hazard (see Sec.\ref{sec:odhhazard}).

 In case of a fire alarm personnel should leave the area. Upon seeing a fire
 or unexplained smoke one should activate the fire alarm, leave the area and
 call 911 and 4444 from a safe place (see \cite{EHScebaf}).
  
 The contacts for Hall C fire safety issues are given in Tab.\ref{tab:ehs:fire}.
 
 \begin{namestab}{tab:ehs:fire}{Contacts for radiation safety issues}{%
   Contacts for radiation safety issues}
  \WalterKellner{\em Safety Warden}
  \BertManzlak{\em EHS Engineer}
 \end{namestab}

\begin{safetyen}{0}{0}
\section{Electrical Hazard} 
\label{sec:electrhazard}
\end{safetyen}

 Almost every subsystem in Hall C requires AC and/or DC power. Due to the high current
 and/or high voltage requirements of many of these subsystems the power supplies providing this 
 power are potentially lethal.

 Aside from the resetting of a small branch circuit breaker you should not attempt to solve
 any other problems associated with AC power distribution without consulting responsible personnel. All
 the power distribution boxes are clearly marked to aid in finding the appropriate circuit breaker in
 the event of a problem.

 There is a ``Hall C power'' crash button in the counting house. This is intended for dire emergency use. 
 It is possible to cause severe damage to Hall C systems
 %(in particular the hadron spectrometer dipole) 
 by inappropriate use of this power kill switch.

 Anyone working on AC power in Hall C must be familiar with the EH\&S\cite{EHScebaf} manual 
 and must contact one of the responsible personnel. Lock and Tag training may also be required.

 The DC power supplies energizing the magnets can provide a very high current.
 There is a danger of metal tools coming into contact with exposed leads, shorting out the
 leads, depositing a large amount of power in the tool, vaporizing the metal, and creating an arc.
 These hazards are mitigated by covers installed around the leads,
 preventing accidental access to them. The covers must not be removed
 unless the magnet is turned off using the Lock and Tag procedure by trained
 personnel. 

 The electronics NIM, CAMAC, FASTBUS and VME crates are equipped with high current
 DC power supplies for $\pm$5~V and other low voltages. Although their power 
 is typically lower than the power supplied for magnets,
 care should be taken to avoid accidental contact to the leads with metal tools.
 Typically, covers are installed on the back of the crates or the racks,
 in order to mitigate the hazard. 

 Another electrical hazard is caused by high voltage (HV) in a range of 1-3~kV  DC power used 
 for photomultiplier tubes. The current per channel, provided by the appropriate power supplies,
 is limited to about 1-3~mA. 
 The HMS and SHMS detectors 
 (see Sec.\ref{chap:detectors}), as well as the beam line equipment use 
 hundreds of such channels. 
 Typically, the power is provided through special cables (of red color). The cables and SHV connectors 
 meet the existing EH\&S standards. Even with the cables disconnected, an accidental
 contact with the power electrodes is not probable. In order to avoid the hazard to the personnel
 as well as damage to the equipment, one should not attach/remove HV cables or the
 phototube bases when HV is present on a given channel. Formally, the ``Lock out / Tag out'' 
 procedure is not required to operate this equipment. However, turning the HV off
 and making sure that it is not accidentally turned on remotely or locally, is required.
The CAEN SY4527 and SYXXX power supply mainframes,
used in Hall C, are equipped with
 a control key on the front panel. The key should be turned to ``local'' mode in order to
 avoid remote operation. If the power supply is located far from the working place,
 it is recommended that the crate be turned off. %and the key be removed.
 
 Numerous cables, including HV cables and high current cables, are installed in trays, 
 racks and other accessible areas. Damage to these cables may result
 in hazards to personnel and equipment. 

 The contacts for Hall C electrical safety issues are given in Tab.\ref{tab:ehs:electrical}.
 
 \begin{namestab}{tab:ehs:electrical}{Contacts for radiation safety issues}{%
   Contacts for radiation safety issues}
  \WalterKellner{\em Safety Warden}
 \end{namestab}

 
\begin{safetyen}{0}{0}
\section{Mechanical Hazard} 
\label{sec:mechhazard}
\end{safetyen}

\textbf{UPDATE FOR HALLC.}
One source of mechanical hazards includes the heavy movable 
 elements in Hall C, like the HRS~\ref{sec:specterometers-safety}.
% and the detector hut doors.
 In order to alert personnel, visible and audible signals are issued when
 the spectrometers
% or the doors
 are moving.   
 Spectrometer motion can be controlled remotely from the counting house,
 in the angular range $>15^\circ$. Motion at smaller angles
 must be performed by the hall technicians only (see \ref{tab:ehs:mechanical}).
 Since motion at large angles may be hindered by equipment stored on the floor,
 Ed Folts provides ``administrative limits'' for spectrometer motion for
 the current time period. Typically, the safe limits for the HRS motion 
 are enforced by pins planted in the hall floor, however the shift crews should
 be aware of the current limits and never exceed them.

 There are conventional
 hazards like fall hazards and crane hazards. The installed safe ladders and hand rails
 mitigate the fall hazards. Working on elevated areas beyond the hand rail protection
 requires the use of safety harnesses or other means. One should consult the contact
 personnel (see Table\ref{tab:ehs:mechanical}) before starting such a work.
 The safety of crane operations is supervised by the hall technical staff. 

 The contacts for Hall A mechanical safety issues are given in Tab.\ref{tab:ehs:mechanical}.
 
 \begin{namestab}{tab:ehs:mechanical}{Contacts for mechanical safety issues}{%
   Contacts for mechanical safety issues}
  \WalterKellner{\em Safety Warden}
 \end{namestab}

 
\begin{safetyen}{0}{0}
\section{Hazard from Strong Magnetic Fields} 
\label{sec:maghazard}
\end{safetyen}

 Personnel working in the proximity of the energized, strong  magnets of the
 HMS, SHMS or the beam line
 are exposed to the following magnetic hazards:
 \begin{list}{--}{\setlength{\itemsep}{-0.cm}}
    \item electrical hazards, described in Sec.\ref{sec:electrhazard};
    \item danger of magnetic objects being attracted by the magnet fringe field, and becoming airborne;
    \item  danger of cardiac pacemakers or other electronic medical devices no longer
           functioning properly in the presence of magnetic fields;
    \item danger of metallic medical implants (non-electronic) being adversely affected
          by magnetic fields.
 \end{list}

 Several measures are taken to mitigate the hazards. 
 Whenever a magnet is energized, a flashing red light on the magnet or on the
 magnet support structure is activated to notify and warn personnel of 
 the associated electrical and magnetic field hazards.
 
 Administrative measures are implemented to
 reduce the danger of magnetic objects being attracted by the magnet 
 fringe field and becoming airborne. (Note that for most magnets strong magnetic 
 fields are only encountered within non-accessible areas inside the magnet.) 
 Areas where these measures are in effect are clearly marked.

 To reduce the danger of magnetic fields to people using pacemakers or other medical
 implants, warning signs are prominently displayed at the entrance to the hall. 

 The contacts for Hall C magnetic safety issues are given in Tab.\ref{tab:ehs:magnetic}.
 
 \begin{namestab}{tab:ehs:magnetic}{Contacts for mechanical safety issues}{%
   Contacts for mechanical safety issues}
  \WalterKellner{\em Safety Warden}
 \end{namestab}

\begin{safetyen}{0}{0}
\section{Cryogenic and Oxygen Deficiency Hazard (ODH)} 
\label{sec:odhhazard}
\end{safetyen}

The superconducting magnets are all operated at temperatures of about 4~K. 
This temperature is obtained by refrigeration with liquid (or super critical) 
helium supplied from the End Station Refrigerator (ESR).

\texbf{Review for Hall C - what is flow rate}
During normal operation the superconducting magnets consume $\sim$14~g/s of Helium. 
In addition, the cryostats of these magnets have an inventory of liquid Helium. If a magnet ``goes
normal'' for whatever reason this Helium inventory will be rapidly boiled. Relief systems have been 
installed on the magnets to protect the vessels from building undue pressures during a quench event. 
However, all superconducting magnets are at least somewhat subject to damage in the event of a quench. 
The magnets have quench protection circuitry designed to safely dispose of the magnets' 
stored electromagnetic energy.

Contact with cryogenic fluids presents the possibility of severe burns (frostbite). 
When handling these fluids, Liquid Nitrogen or Helium, one must follow the
guidelines in the EH\&S manual\cite{EHScebaf}. These guidelines mandate the use 
of cryogenic gloves and eye protection.

All volumes in the cryogenic systems which can be isolated by valves or any other 
means are equipped with pressure relief valves to prevent explosion hazards.
The release and subsequent expansion of cryogenic fluids presents the possibility 
of an oxygen deficiency hazard. Rapid expansion of a cryogenic fluid in a confined 
space presents an explosion hazard. Cryogenics in Hall C are present in the superconducting 
HMS and SHMS  magnets, and the scattering chamber with its cryogenic targets. The total 
inventory of cryogens in the magnets and targets present a minimal ODH hazard in all 
areas of the hall except the area above the Hall C crane\infolevone{ (see also Sec.\ref{sec:targ:odh})}.

There are a number of vessels which are normally filled with oxygen free atmosphere. 
These include the gas Cherenkov, the spectrometer vacuum space and the scattering chamber. 
Service of these vessels could represent a ODH (confined space) hazard.

Hall A is listed as an Oxygen Deficiency Hazard area of Class 0. 
No unescorted access is allowed without up-to-date JLab ODH training.

No one should enter the Cherenkov tanks while there is gas inside these tanks. 
The tanks should be pumped out and filled with air before access to the interior 
of these tanks is permitted. 
The HMS and SHMS detector huts may present an ODH hazard in case of the fire suppression 
system activation (see Sec.\ref{sec:firehazard}), if the doors of the hut are shut. 
No one may stay in the huts with the doors shut.
  
 The contacts for Hall C cryo and ODH safety issues are given in Tab.\ref{tab:ehs:cryo}.
 
 \begin{namestab}{tab:ehs:cryo}{Contacts for cryogenic safety and ODH issues}{%
   Contacts for cryogenic safety and ODH issues}
  \BertManzlak{\em EHS Engineer}
  \WalterKellner{\em Safety Warden}
 \end{namestab}

\begin{safetyen}{0}{0}
\section{Vacuum and High Pressure Hazards} 
\label{sec:vachazard}
\end{safetyen}

\textbf{UPDATE below for HALL C}
The greatest safety concern for the vacuum vessels in use in Hall C are the thin 
aluminum, titanium or kapton windows that close the entrance and/or exit of 
these vessels. The HRS spectrometer vacuum vessel, and the Hall A Scattering Chamber 
both contain thin windows.

The HRS Vacuum System is described in detail in Chapter~\ref{chap:vacuum}. 
The space between the magnet poles of both spectrometers is evacuated in order 
to diminish multiple scattering. The entrance and exits of the main spectrometer 
volumes are covered by relatively thin vacuum windows. 
The vacuum safety of the cryo-target is described in Sec.\ref{sec:target-cryo-safety}. 

The HRS spectrometer vacuum can has a volume of approximately 6 m$^3$.
%, representing a stored energy of 6 \Theta  105 Joules. 
The circular (7 inch diameter) entrance window on the front of Q1 is made of 0.007~inch thick 
kapton while the rectangular 90.89" by 6.41" exit window located in the shield hut below the VDC 
is 0.004~inch thick Ti (3,2.5) Alloy. The scattering chamber contains two windows 
constructed from 0.016~inch thick 5052 aluminum.

Installation of vacuum windows can only be done by the responsible personnel following
detailed instructions provided in the Operations Manual.
Before entering the detector huts or pivot area, all personnel should
check the spectrometer and/or scattering chamber vacuum gauges. If the spectrometers 
and/or scattering chamber are under vacuum:
 \begin{list}{--}{\setlength{\itemsep}{-0.cm}}
    \item Use careful judgment if it is necessary to work near the vacuum windows;
    \item Do not work near the windows any longer than is absolutely necessary;
    \item Never touch the vacuum windows, neither with your hands nor with tools;
    \item Do not place objects so that they may fall on the windows, etc.;
    \item Hearing protection is required when working near the target chamber windows and 
          is recommended in the shield huts.
 \end{list}

Window covers must be in place over the target chamber whenever the hall is in
Restricted access. The windows covers must also be employed whenever extensive 
work must be done in the area of the pivot.
Window covers must be placed over the spectrometer exit windows when they are
not covered by the detector package.

The highest gas pressure used in Hall C is about 2000~PSI ($\sim$140~atm) in the bottles
of argon and other gasses used to flush the drift chambers of HRS 
(see Chapter~\ref{chap:hrs-det-gas}). The bottles (``cylinders'') are installed
in a gas shed outside the hall. To ease handling, gas bottle carts are available for use 
in the hall and the gas shed. Typically, the Hall C technicians handle the gas bottles. 
Bottles must never be left free standing. They must always be stored in a rack, 
on a cart or tied to a support.

Polarized $^3$He targets contain Helium gas at $\sim$10~atm in a glass cell.
Hearing protection is mandatory when working near the glass 
cell\infolevone{ (see Sec.\ref{sec:targ-hel-saf})}.   
 
 The contacts for Hall C vacuum and pressure safety issues are given in Tab.\ref{tab:ehs:vacuum}.
 
 \begin{namestab}{tab:ehs:vacuum}{Contacts for vacuum and pressure issues}{%
   Contacts for vacuum and pressure issues}
  \BertManzlak{\em EHS Engineer}
  \WalterKellner{\em Safety Warden}
 \end{namestab}

\begin{safetyen}{0}{0}
\section{Toxic Materials Hazard} 
\label{sec:toxichazard}
\end{safetyen}

 Some of our target materials may pose a safety concern. Presently the
 only hazardous target materials used is ceramic Beryllium-Oxide (BeO).
 In solid form, BeO is completely safe under normal conditions of use.
 The product can be safely handled with bare hands. However, in powder form 
 all Beryllia are toxic when airborne. Overexposure to airborne Beryllium particulate 
 may cause a serious lung disease called Chronic Berylliosis. 
% Beryllium has also been listed as a potential cancer hazard. 
% Furthermore exposure to Beryllium may aggravate medical conditions related to 
% airway systems (such as asthma, chronic bronchitis, etc.). 
 Since beryllia are mainly dangerous in powdered form, do not machine, break, or 
 scratch these products. Machining of the Beryllia can only be performed after 
 consulting the EH\&S staff. It is good practice to wash your hands after handling 
 the ceramic BeO.

 Lead shielding blocks and sheets are also potentially toxic. Always wear gloves 
 when handling lead, unless it is completely painted or wrapped in Heavy-Duty 
 Aluminum Foil. Do not machine lead yourself, contact the EH\&S personnel or 
 the Jefferson Lab workshop to ask for assistance prior to machining lead. There 
 are lead storage areas designated in the hall, when not in use, shielding should 
 be stored in an area marked for lead storage.

 The Material Safety Data Sheets (MSDS) for all materials encountered in the workplace 
 are available. If in doubt ask the hall safety warden or contact the physics division 
 EH\&S staff.

 The contacts for Hall C material safety issues are given in Tab.\ref{tab:ehs:toxic}.
 
 \begin{namestab}{tab:ehs:toxic}{Contacts for material safety issues}{%
   Contacts for material issues}
  \BertManzlak{\em EHS Engineer}
  \WalterKellner{\em Safety Warden}
 \end{namestab}

\obsolete{
 % aaa
} 

