\section{The Hall~C Scattering Chamber}

The scattering chamber consists of a large central band made from a single
forged ring of 6061-T6 Al. This ring has an inner diameter of
48.5 inches and a  2.5 inch wall (0D = 53.5 inch). The ring has
cutouts consistent with the vertical angular acceptances and the
full angular ranges of the two Hall~C spectrometers. In addition,
the cutout for the SOS has been made to accommodate 20 degrees
of out of plane movement. There are also openings through which the beam
will enter and exit, a pumping port, several ports for viewing, and
some ports for as yet unspecified purposes.

There are two shells which are attached to the outside of the central
band. These are cylindrical sections which have an inner diameter
equal to the outer diameter of the scattering chamber. These shells
are designed to carry sliding vacuum seals so that the scattering
chamber vacuum can be directly coupled to that of the two spectrometers.
The SOS shell is mounted on a set of ``sliders" and can be moved
up and down by a series of linear actuators in order to accommodate the
out of plane motion of the spectrometer.

In the early stages of operation these shells are replaced by
clamps which hold thin fixed metal vacuum windows.

The opening for the SOS is five inches tall. The metal window is
5052-H39 aluminum 0.008 inches thick. This is the same foil that was used to
cover the five inch tall window on the temporary Hall~C scattering chamber
(The SLAC chamber). This is the same material that was used by SLAC when
the chamber was in use there.
The opening for the HMS
is eight inches tall. This will be spanned by a window of
5052-H34 aluminum 0.016 inches thick. The tank has been pumped down with
{\bf both openings covered by 0.008 inch thick 5052-H39 }
and vacuum cycled several
times. The crinkling pattern was examined and the inter crinkle spacing
was used as input for stress calculations. In this approximation the
total window is treated as a collection of smaller windows each of which
has a height equal to that of the full opening and a width given by the average
inter crinkle spacing. These calculations indicate that the windows
proposed above (0.008 inch for SOS and 0.016 inch for HMS) have at least
a factor of two safety margin.

These windows have been tested to failure with over pressure from the
outside.  The average failure pressure exceeded 40 psid and thus they
have a safety margin of over 2.5.  

There are also top and bottom plates which complete the main body of the
scattering chamber.

The bottom plate allows the chamber to be mounted to the
solid shaft which forms the pivot axis for the two spectrometers.


The top plate has a number of openings. The largest of these allows
the cryotarget plumbing and lifting mechanisms into the vacuum and is sealed by
a large diameter bellows. There is also a three inch diameter tube through
which the solid targets are inserted.
The other three openings are capped by eight inch diameter aluminum
conflats and currently have no specified function.

The beam entry and exit tubes are vacuum coupled to the scattering chamber with
metal seals. The beam exit tube terminates in a 0.015 inch thick beryllium foil.
This window should be regularly inspected for signs of deterioration.

%All questions about the scattering chamber or its vacuum windows should
%be referred to ???
