%--- Description of components
% Need to setup a default online analyzer location for checking data quality
% Add note on how to run histogrammer and provide some sample data
% quality histograms

\newcommand{\cppaver}{1.5.24}

\infolevone{

The standard offline analysis software for Hall C data is ``hcana''
(a.k.a.\ the ``Hall C Analyzer''),
an object-oriented C++ class package developed
by Hall C users and staff.  hcana
is built on top of the Hall A analysis software ``podd'' and uses the
ROOT~\cite{ROOTcern} programming framework, developed at CERN.
All of ROOT's analysis and visualization tools are available from within hcana,
plus specialized classes for Hall C detectors and physics analysis.
More information about the
software (downloads, documentation, etc.) can be found at
\begin{center}
  \url{https://hallcweb.jlab.org/wiki/index.php/ROOT_Analyzer}.
\end{center}

hcana is newly developed for the 12 GeV era and supercedes the Fortran
based ENGINE used in the 6 GeV era for analysis of HMS and SOS data.
This new analysis software curently replicates the algorithms of the
ENGINE, so in principle hcana could be used to analyze 6 GeV era data and
reproduce the results of the Fortran analyzer.

Individual analysis components are designed as plug-in modules
that can be loaded dynamically from an analysis script or otherwise as needed.
As a result, the scope of the data analysis is
largely user-configurable. Only data from the configured spectrometers and
detectors are analyzed, and only the user specified physics
calculations are carried out. Configuration can occur at run
time without any need for recompilation of the program.

Currently supported are the analysis of the Hall C HMS, SHMS (and SOS)
spectrometers,
%the beamline instrumentation,
scaler and EPICS~\cite{EPICSwww}
slow control data.  %, and beam helicity information.
The event decoder is compatible with the CODA~\cite{CODAwww}
event data format described in the section on Data Acquisition.
%Decoding of basic helicity information as well as a sophisticated
%algorithm for decoding and prediction of the G0 helicity sequence is
%possible.
The following detectors can be used in the spectrometers:
\begin{itemize} \setlength{\parskip}{0ex}
  \item Horizontal Drift Chambers (DCs)
  \item Hodoscopes (one or more planes of scintillator or quartz bars)
  \item Gas Cherenkov counters
  \item Aerogel counter
  \item Shower counters (one or more planes of side readout bars and
    optionally a fly's eye array.)shower, preshower, pion rejectors with
\end{itemize}
\infolevone{

The Drift Chamber code performs tracking in the focal plane and is able to
identify multiple tracks.  The THcHallCSpecrometer apparatus class
projects these tracks back to the target, determining momentum, target
position, and scattering angles for each track.  This class also
selects a best or ``golden'' track using various criteria including
track fit $\chi^2$ and information from hodoscopes.

The scintillator, Cherenkov, and shower counter classes perform basic decoding,
calibration (offset/pedestal subtraction, gain multiplication), and summing
(for Cherenkovs) or cluster-finding (for showers) of hits.
%The cluster-finding algorithm of the shower class is basic and currently
%only capable of finding a single cluster per event. These classes are largely
%generic and should be able to accommodate most new detectors of the respective
%type, even with a different geometry and number of channels.
% Not so generic for Hall C

Several beamline apparatuses are available: a dummy beam (always at zero
position and angle), an ``unrastered'' beam, and a ``rastered'' beam.
BPM and raster detectors are implemented and
can be analyzed to obtain the beam position
on an event-by-event basis.  The BPM code currently only supports
standard ADCs (e.g.\ LeCroy), not the older Struck readout.

%Plug-in libraries have been developed by users for the following equipment:
%the BigBite spectrometer (in several configurations),
%the neutron detector of the $G_E^n$ experiment (E02-013),
%the focal-plane polarimeter (FPP) and the ring-imaging Cherenkov counter
%(RICH), which are optional detectors for the HRS, and the DVCS photon
%detector. Most of these libraries are available from the Podd web page
%cited above.

To carry out standard post-reconstruction calculations, the following
so-called physics modules are available:
\begin{itemize} \setlength{\parskip}{0ex}
  \item Single-arm electron kinematics ($Q^2$, $\omega$ etc.)
  \item Coincidence kinematics (missing energy etc.)
  \item Deuteron photodisintegration kinematics
  \item Single-arm elastic scattering kinematics from detected recoil
          particle
  \item Coincidence time
  \item Reaction point (vertex position) reconstruction
%  \item Extended target tracking corrections for the HRS
  \item Energy loss corrections
\end{itemize}
There is no limit (other than machine resources) to the number of physics
modules that can be configured for an analysis run.
Identical modules can be added multiple times with different parameters.
For example, one can calculate ``electron kinematics'' using both uncorrected
and energy-loss corrected tracks in one analysis pass without modifying
or recompiling any code. For experiments requiring
specialized calculations of kinematics or any other quantities,
writing a new physics module is the preferred approach.

The results of calculations performed by the various analysis modules
(spectrometers, detectors, physics modules) are made available via
so-called global variables.
Global variables provide access to data via names (text strings).
Scalars as well as fixed and variable-size arrays are supported.
The global variable names are used in the definition of the analysis output
and of logicals.

Tests and cuts (``logicals'') can be defined
dynamically at run time. They can be used for controlling the analysis flow,
keeping statistics, pre-computing logical values to be included in the output,
or applying cuts to histograms written to the output.
If certain tests fail for a given event, further analysis of that
event can be skipped, and the event is not written out.  Such tests
can be put at the end of all the major stages of the analysis.  This
allows making decisions about an event early in the analysis,
improving performance.  A summary of all test results is written to a
file at the end of the analysis.

Data of interest that is save in global variables
can be histogrammed and/or written to
a ROOT Tree in the output file.  The contents of the output
is defined dynamically at the beginning of the analysis.
Both 1- and 2-dimensional histograms are supported. Histograms
can be filled selectively using logical expressions (cuts).

Text summaries of run and analysis information can be made with the
PrintReport method of the THcAnalyzer class.  Using text based
templates, similar to templates used in the old ENGINE analyzer,
hardware scalers, parameters, statistics of cut success, and other
information such as EPICS values can be written to ``reports.''

} %end \infolevone

\infolevfour{
Table~\ref{offl:tab:modules} lists the analysis modules available in
version \cppaver\ of Podd.
%The global variables available
%from these modules are described in Table~\ref{offl:tab:variables}.

% Tables of modules

\begin{table}[p]
\begin{tabular}{|l|l|}
\hline
Class name & Description \\
\hline
\multicolumn{2}{|l}{Support} \\
\hline
THcAnalyzer & Hall C customized analyzer event loop \\
THcParmList & Read and store detector and physics parameters \\
THcFormula & Extends ThaFormula to allow access to parameters and cut
statistics \\
\multicolumn{2}{|l}{Apparatuses} \\
\hline
THcHallCSpectrometer  & Hall C spectrometer (SHMS, HMS or SOS) \\
THaIdealBeam & Dummy beam with zero position and angle. \\
THcRasteredBeam & Beam with raster \\
THaUnrasteredBeam & Beam without raster (for calibration) \\
THaDecData   & Miscellaneous decoder raw data \\
\hline
\multicolumn{2}{|l}{Detectors} \\
\hline
THcDC          & Drift Chamber Package \\
THcHodoscope   & Set of Scintillator (or quartz) planes \\
THcCherenkov    & Gas Cherenkov \\
THcAerogel      & Aerogel Cherenkov \\
THcShower       & combination of preshower and shower \\
%THaBPM          & beam position monitor with standard ADCs \\
THcRaster       & beam raster system \\
%THaHelicity     & beam helicity information (in-time or delayed) \\
\hline
\multicolumn{2}{|l}{Physics Modules} \\
\hline
THaReactionPoint & vertex position (intersection of spectrometer track \\
                 & with beam) \\
THaTwoarmVertex  & vertex position (intersection of two spectrometer tracks) \\
THaAvgVertex     & vertex position (average of reaction points from two \\
                 & spectrometers) \\
THaElectronKine  & single-arm electron kinematics \\
THaPrimaryKine   & single-arm kinematics for particle with arbitrary mass \\
THaSecondaryKine & coincidence kinematics \\
THaPhotoReaction & deuteron photodisintegration kinematics \\
THaSAProtonEP    & elastic $A(e,A)e'$ kinematics from detected $A$ \\
THaExtTarCor     & extended target corrections \\
THaCoincTime     & coincidence time calculation \\
THaS2CoincTime   & coincidence time using S2 scintillator in HRS \\
THaGoldenTrack   & selects Golden Track from multiple reconstructed tracks \\
THaDebugModule   & prints values of global variables for each event and waits\\
\hline
\end{tabular}
\caption{Analysis modules available in Podd.  Not tested with hcana.}
\label{offl:tab:modules}
\end{table}

% Tables of global variables
%\begin{table}
%\vspace*{1cm}
%\caption{Global variables available in version \cppaver\ of Podd}
%\label{offl:tab:variables}
%\end{table}

} %end infolevfour


%--- Procedures
\obsolete{ % Comment out until in better shape
%\infolevtwo{
\section{Running hcana}
% Need to define where precompiled versions on DAQ machine are.
Precompiled binaries of the latest version of the Podd are
installed on the Hall A counting house analysis machines
(adaql1--10). To run Podd, log into any
standard account on these computers, for example as \mycomp{adaq},
and type\\

\noindent\mycomp{analyzer}
\vspace{2ex}

As installed, Podd uses the default database in
the location pointed to by the environment variable \mycomp{DB\_DIR}.
The Hall A staff makes an effort to keep this database reasonably
up-to-date for completed experiments. If you wish to analyze older data,
the default database might work for you. However,
if you wish to use a customized set of database files
specific to your experiment (usually the case for the current experiment),
you will need to re-define \mycomp{DB\_DIR} to
point to the location of that database before starting the Podd.
For details on the database, see Section~\ref{offl:sec:database}.

The pre-installed Podd may not work if
the \mycomp{PATH} and/or \mycomp{LD\_LIBRARY\_PATH} variables have been
changed from the system defaults. If this is the case, you should
correct the login script(s) of the problematic account.
To restore the system defaults, you may execute one of the following
commands:

\begin{tabbing}
For csh/tcsh shells:$\qquad$ \= \mycomp{source /adaqfs/apps/env/login.adaq}\\
For bash shells:     \> \mycomp{source /adaqfs/apps/env/profile.adaq}
\end{tabbing}

If Podd is to be used outside of the Hall A counting house
environment, it is currently necessary to build the program from source.
A {\tt tar} archive of the sources can be obtained from the following
location
\begin{center}
  \url{http://hallaweb.jlab.org/podd/download}.
\end{center}

\section{Preparing Analysis of a New Experiment}
\label{offl:sec:newexpt}
Setting up offline analysis for a new experiment typically
involves the following steps:
\begin{enumerate}
  \item Determine the experimental configuration (spectrometers,
        detectors, beamline) to be analyzed and identify the corresponding
	analysis modules. \label{offl:item:step1}
  \item Create a database for the new experiment, using the
        start date of the data taking as the time-stamp for new
	new entries (see Section~\ref{offl:sec:database}).
	At the minimum the database should contain up-to-date
	detector map entries for every detector and rough starting
	values for the spectrometer reconstruction matrix elements
	and VDC timing offsets. Often this information can be
	carried over from a previous experiment with only minor
	modifications. Also, enter any other calibrations and geometry
	data that are available, even if approximate.
  \item In the database, create initial run database values.
        These are typically the starting beam energy and spectrometer
	momentum and angle settings. If the experiment is already completed,
	extract the history of these settings from logs and enter them
	into the database. These values affect the kinematics calculations;
	they are not important for detector checkout.
  \item Determine which physics calculations are needed for the
        offline analysis and identify corresponding Physics Modules.
	\label{offl:item:step3}
  \item Identify desired output histograms and tree variables.
        Create an output definition file. The file
        \mycomp{\$ANALYZER/examples/output\_example.def} contains
	most of the necessary documentation.
  \item If desired, create a definition file for logicals.
        An example generating detailed VDC statistics is
	given in \mycomp{\$ANALYZER/examples/cuts\_example.def}.
  \item Write a CINT\footnote{CINT is a C/C++ interpreter that acts
        as the interactive interface to ROOT and Podd.}
        script that sets up the configuration identified in
	Step~\ref{offl:item:step1} and the physics analysis decided on
	in Step~\ref{offl:item:step3}. Often, a script from a
	previous experiment, or one of the examples in
	the directory \mycomp{\$ANALYZER/examples}, can serve as a guide.
	The script usually also locates raw data files, creates
	one or more \mycomp{THaRun} objects, configures various
	options of the event loop object \mycomp{THaAnalyzer}, and
	starts the replay. In particular, the names of the output
	file, the output definition file, and the logicals definition
	file must be given to \mycomp{THaAnalyzer}.
  \item Identify the plots that you wish to generate from the analysis
        results and write a script to create them. This may be part
	of the script created in the previous step.
	Note that there is no need to quit Podd and start a new
	session or another program after completion of the analysis; all of
	ROOT's visualization tools are available from within Podd.
\end{enumerate}

\section{Parameter Database Files}
\label{offl:sec:database}
hcana is designed to use the same style of parameter files that were
used for the Fortran ENGINE.

Version \cppaver\ of Podd uses simple ASCII text files
to store database information. There is usually one file for each
analysis module. The name of each file is composed as follows:\\

\noindent \mycomp{db\_apparatusname.detectorname.dat}
\vspace{2ex}

For example, a Cherenkov detector named ``a1'' which is part of the
Left HRS spectrometer, named ``L'', would be associated with
a database file named \mycomp{db\_L.a1.dat}.

The ``run database'', which contains global run-specific parameters
such as beam energy and spectrometer momentum and angle settings,
is stored in a special file named \mycomp{db\_run.dat}.

All of the above database files should be stored in a location
that can be modified by the user, for instance in \mycomp{$\tilde{}$/DB}.
{\bf The environment variable \mycomp{DB\_DIR}
must be defined to point to this top-level database
directory.} Since database parameters change with time, database files
are organized in time-dependent subdirectories within \mycomp{\$DB\_DIR}.
The name of each subdirectory has the form \mycomp{YYYYMMDD}, where
\mycomp{YYYY}, \mycomp{MM} and \mycomp{DD}
represent the year, month, and day, resp., of the
date that is the start of the validity of the entries. Upon initialization,
Podd locates the most appropriate time-dependent subdirectory based on
the contents of \mycomp{\$DB\_DIR}
and the time-stamp of the run to be replayed.
Often there is only one time-dependent subdirectory per experiment, but
if significant changes occur during an experiment, it may be appropriate to
create several directories.
A finer division of time-dependent information
can be provided by timestamps within each database file. This
is especially true for the run database file which frequently
will have many time-stamped sections.

For example, an experiment running in April and May of 2004 would
create a database subdirectory \mycomp{$\tilde{}$/DB/20040401} and set
\mycomp{DB\_DIR=$\;\tilde{}$/DB}. Other files supporting the replay
of this experiment would reside in an experiment-specific
directory, usually \mycomp{\$EXPERIMENT}.


} %end \infolevtwo

%--- Principles of operation
\infolevthree{

\section{Program Design Overview}
Spectrometers (and similar major installations) are abstracted in an
Apparatus class hierarchy, while individual detectors belong to a
Detector class hierarchy. Apparatuses are collections of detectors
that are analyzed in a particular way. Specialized physics analysis,
such as kinematics calculations, vertex determination, and energy loss
corrections, can be done in Physics Modules.  All three types of
objects, Apparatuses, Detectors, and Physics Modules, are kept in
lists that are processed during replay. In setting up the replay, it
is up to the user which objects to place in the lists.

%\infolevfour{
%The class hierarchies of Apparatuses and Detectors are shown in
%Figures~\ref{offl:fig:apparatus-hierarchy}
%and \ref{offl:fig:detector-hierarchy}.
%}
Both the Apparatus and the Detector class hierarchies,
as well as the Physics Modules, inherit from a common base class,
\mycomp{THaAnalysisObject}.
%describe details of hierarchies
Physics Modules currently do not use a
particular class hierarchy; all physics modules inherit from
\mycomp{THaPhysicsModule}, which in turn inherits from
\mycomp{THaAnalysisObject}.

The behavior of existing analysis modules can be modified or extended
easily by using class inheritance. In such a case, the only code that
needs to be written is the implementation of the new feature.  For
example, the standard Cherenkov detector class currently only
calculates the total sum of ADC amplitudes. For a new type of
Cherenkov counter, or to do a more sophisticated analysis of the
standard Cherenkov detectors, one might want to calculate separate ADC
sums for certain groups of PMTs.  To do so, one would write a new
class inheriting from the standard Cherenkov class, which could
contain as little as one function, performing the additional
calculations, and the corresponding data members.  New types of
detectors and even entire spectrometers, as well as new types of
physics calculations, can be added similarly easily, again using class
inheritance.  No change to and no rebuilding of the core program is
necessary to support such new modules.

} %end \infolevthree

%--- Large figures
\infolevfour {

% Class hierarchy diagrams

%\begin{figure}
%\includegraphics[width=\textwidth]{apparatus-classes}
%\caption{Class hierarchy for Apparatuses in version \cppaver\ of Podd}
%\label{offl:fig:apparatus-hierarchy}
%\end{figure}
%
%\begin{figure}
%\includegraphics[width=\textwidth]{detector-classes}
%\caption{Class hierarchy for Detectors in version \cppaver\ of Podd}
%\label{offl:fig:detector-hierarchy}
%\end{figure}
} %end \infolevfour

\begin{safetyen}{10}{15}
\subsection{Responsible  Personnel}
\end{safetyen}
The responsible personnel are shown in table \ref{tab:offline:personnel}.
\begin{namestab}{tab:offline:personnel}{Offline analysis:
authorized personnel}{%
      Offline analysis: authorized personnel.}
  \MarkJones{\em Contact}
%  \StephenWood{}
  \OleHansen{}
\end{namestab}
}
