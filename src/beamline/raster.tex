
% Operations Manual Title
\infolevone{\section{Raster Systems}}
%\setcounter{subsection}{0}

%%% The following lifted from the Hall A manual

% ESAD Title
\begin{safetyen}{0}{0}
\infolevone{\subsection{Safety Information}}
\infoleveqnull{\section{Raster Systems}}

The Hall C beamline includes two raster systems permanently installed.
These systems, the M\o ller Raster (MR) and the Fast Raster (FR),
rapidly move the beam position on the M\o ller Polarimeter target and
the Hall~C target, respectively, in order to prevent possible damage
to the target by overheating. A third raster system (the slow raster)
can be installed for use with polarized targets and is not part of the
``standard'' beamline. The M\o ller raster is only used in special
circumstances and operated under the auspices of a test plan or
TOSP. The FR serves to reduce boiling and uncertainties in target
thickness for cryogenic targets.  The beam rastering is done with with
vertical and horizontal air-core magnets that are driven by AC
currents from a 250 W audio amplifier. The fast raster coils are
shielded with a Plexiglas cage and a ``Magnet On'' sign indicates the
presence of current in the raster coils.
%\sawnote{Is it still a 250W audio amplifier?}
No DC fields exist when these magnets are operated and no remnant
field remains when the current is switched off.

The M\o ller raster magnets are located on the accelerator side of the
beamline shield wall (on the 3C07 girder) so are not normally
accessible from Hall C.

The following conditions lead to a fast shutdown of the raster devices:
\begin{itemize}
\item{Crate power failure}
\item{Magnets power failure}
\item{Overcurrent detection (short occurs inside the magnets)}
\item{Over temperature}
\item{Detection of missing cycles or improper frequency}
\item{VME system reset}
\item{Phase-lock network is broken}
\end{itemize}

\noindent The FR power drivers have an automatic fault display and
shutdown. The signals are also sent to FSD.

The following Reference Documents exist:
\begin{itemize}
\item{Requirement for beam raster monitor for the beam dump and target,
R. C. Cuevas, C. Yan, 12 July, 1995}
\item{Technical requirement for Hall~C beam dump raster,
C. Yan, June 24, 1995, JLab-R-94-02}

\item{M\o ller Raster Manual~\cite{docdb:moellerraster}}
\end{itemize}

\infolevone{\subsection{Hazards}}
\infoleveqnull{\subsection{Hazards}}

The primary hazards associated with the updated fast raster system
are electrical and are addressed in the accompanying OSP.

\infolevone{\subsection{Mitigations}}
\infoleveqnull{\subsection{Mitigations}}

The electrical hazards are mitigated by enclosing the raster coils in
Plexiglas covers while the raster power supplies are stored in a
protective enclosure nearby. Only qualified personnel should access
the power supplies or coils. The mitigations of this updated system
are addressed in the accompanying OSP.

\infolevone{\subsection{Responsible Personnel}}
\infoleveqnull{\subsection{Responsible Personnel}}

Points of contact for the fast raster system are listed in the Tab.~\ref{tab:raster:personnel}.
\begin{namestab}{tab:raster:personnel}{Fast Raster responsible personnel}{%
          Fast Raster responsible personnel}
 \MarkJones{Primary Contact}
 \WilliamGunning{Secondary Contact}
\end{namestab}
\end{safetyen}

\obsolete{
\subsection{Details of Equipment}

The beam is rastered on target with an amplitude of
several millimeters at 25 kHz to prevent overheating.  
The raster is a set of four of air-core dipoles located
approximately 14 m upstream of the target. 
Two dipoles are for horizontal (X) motion and
another two for vertical (Y).  During the 6 GeV era
there was only one pair of X and Y, but we have doubled
the raster to account for the energy increase to 11 GeV.
The arrangement along the beamline along the 
direction of the beam will be XYXY.

For a typical 40A current in the raster coils, the
deflection by one pair (e.g. the X direction) of coils, 
in radians, is $\theta = 1.94 \times 10^{-3}/ E$
where $E$ is the electron's energy in GeV.
For example, at $E = 6$ GeV, a 0.32 mrad deflection is achieved.
Projected onto the target (about 14 m away) this is a $\pm$ 4.5 mm
excursion. In Hall C, there are no magnetic focusing elements
between the fast raster and the target so setting the raster
to the appropriate size is rather straightforward.

Since 2003 we've used the triangle-wave 
raster pattern designed by Chen Yan.  
This achieves a very uniform rectangular
density distribution of beam on the target 
by moving the beam with a time-varying dipole
magnetic field whose waveform is triangular
with very little dwell time at the peaks.  
The electronics design is an ``H-bridge''
in which switches are opened and closed 
at 25 kHz, to switch between two directions 
of current (100 A peak-to-peak) 
through the raster coils.

Three new features during the 12-GeV era are 
1) the driver of the H-bridge electronics is now
an Agilent model 33522A waveform generator; and
2) The two X are synchronized with each other, and
the two Y are synchronized.  This makes the kicks
add and allows us to accommodate the higher energy
of the beam; and 3) The entire raster can
be synchronized to an external 10 MHz wave-train
supplied by the polarized injector electronics.
This makes the nominal 25 kHz an exact multiple of
the helicity-flip rate, which achieves a cancellation
of raster noise, important for parity-violation 
experiments only.
The synchronization of the pairs of X and Y are
accurate to within a few nsec.

For most users, these three new features will not be
noticeable and the raster will appear to function
the same as during the 6 GeV running.
A user can view the 
status of the raster in the
EPICS overview screen called ``General Accelerator
Parameters'' where the set-point for the radius amplitude
and the readback of the peak-current in the raster are displayed.

Control of the raster is done by asking the MCC
operators to set up the raster for a particular size
typically 2 mm square.

On occasion, it is desirable to verify the calibration
of the fast raster size vs. set currents. This can be accomplished
by either taking a harp scan with the raster on, or by use
of triggered events from a target with a known width and/or height and
varying the raster size until the edges of the target are seen.


}

%%% The following from the old Hall C manual

%\subsection{Raster Systems}
%
%Vertical and horizontal air-core magnets provide beam
%rastering patterns on the Hall~C target (Fast Raster or FR), the
%Hall~C M\o ller Polarimeter target (M\o ller Raster or MR), and the Hall~C 
%Polarized Target Raster (SR) in order to prevent possible damage to the material by
%overheating. All are AC magnets. No DC fields 
%exist when they are operated. No remnant field remains after
%the current is switched off.
%
%Both FR and SR magnets are driven by a 250 W audio amplifier. The FR
%and SR magnets are shielded by a Plexiglas cage and are marked with a
%``high voltage dangerous'' symbol.  A red light indicates the
%operation of these two raster systems. During their operation, personnel
%should be away from the resonance loop because the rms voltage across
%the magnets is over 400 V.
%
%}%\infolevone
