\infolevone{
\section{Beam Current Measurement}
%\setcounter{subsection}{0}
}

%%% This is lifted from the Hall A manual

The Beam Current Monitor (BCM) is designed for stable, low noise, non-intercepting 
beam current measurements. It consists of an Unser monitor, two rf cavities, 
the electronics and a data acquisition system. The cavities and the Unser monitor 
are enclosed in a box to improve magnetic shielding and temperature stabilization.
The box is located 25 m upstream of the target. You can recognize it as a grey 
object on the stands, about 2 m downstream from where the beam enters the 
hall. 

\infolevone{
  The DC 200 down-converters and the Unser front end
  electronics are located in Hall A. The temperature controller, the
  Unser back end electronics and its calibration current source,
  cavity's RF unit (housing the RMS-to-DC converter board) and all
  multi-meters, VME crate and computers are located in Hall C control
  room.

\subsection{System Layout}

The schematic diagram of the BCM system is presented in
Fig.~\ref{fig:halla_bcm}.
\begin{figure}[htp]
\begin{center}
%\includegraphics[angle=0,width=0.9\textwidth,clip]{habcm_r}
{\linespread{1.}
\caption[Beam Current Measurement: Schematic]{Schematic of the Hall C beam
current measurement system.}
\label{fig:halla_bcm}}
\end{center}
\end{figure}

The Unser monitor is a Parametric Current Transformer designed for
non-destructive beam current measurement and providing an absolute
reference. The monitor is calibrated by passing a known current
through a wire inside the beam pipe and has a nominal output of 4
mV/$\mu $A. It requires extensive magnetic shielding and temperature
stabilization to reduce noise and zero drift. As the Unser monitor's
output signal drifts significantly on a time scale of several minutes,
it cannot be used to continuously monitor the beam current. However,
this drift is measured during the calibration runs (by taking a zero
current reading) and removed in calibrating the cavities.  The more
stable cavities are then used to determine the beam current and charge
for each run. We also use the OLO2 Cavity Monitor and the Faraday Cup
2 at the Injector section to provide an absolute reference during
calibration runs.

The two resonant rf cavity monitors on either side of the Unser
Monitor are stainless steel cylindrical high Q ($\sim 3000$)
waveguides which are tuned to the frequency of the beam (1.497 GHz)
resulting in voltage levels at their outputs which are proportional to
the beam current. Each of the rf output signals from the two cavities
are split into two parts. One part of the signal is converted to 10
kHz signals (by the ``downconverters'') and fed into an RMS-to-DC
converter board consisting of a 50 kHz bandpass filter to eliminate
noise, amplified and split to two sets of outputs, which after further
processing are recorded in the data stream. These two paths to the
data stream (leading to the sampled and integrated data ) will now be
described. (The other part of the split signal is downconverted to 1
MHz signals and represents the old system (pre Jan 99). Only the
HAPPEX collaboration presently uses these signals.)

For the sampled (or EPICS~\cite{EPICSwww} or Slow) data, one of the
amplifier outputs is sent to a high precision digital AC voltmeter (HP
3458A). Each second this device provides a digital output which
represents the RMS average of the input signal during that second.
The resulting number is proportional to the beam charge accumulated
during the corresponding second (or, equivalently, the average beam
current for that second). Signals from both cavity's multi-meters, as
well as from the multi-meter connected to the Unser, are transported
through GPIB ports to the HAC computer where they are recorded every 1
to 2 seconds via the data-logging process which is described in the
calibration procedure. They are also sent through EPICS to CODA and
the data stream where they are recorded at quasi-regular intervals,
typically every two to five seconds.

For the integrated (or VTOF or Fast) data, the other amplifier output
is sent to an RMS-to-DC converter which produces an analog DC voltage
level. This level drives a Voltage-To-Frequency (VTOF) converter whose
output frequency is proportional to the input DC voltage level. These
signals are then fed to Fastbus scalers and are finally injected into
the data stream along with the other scaler information.  These
scalers simply accumulate during the run, resulting in a number which
is proportional to the time integrated voltage level and therefore
more accurately represents the true integral of the current and hence
the total beam charge. The regular RMS to DC output is linear for
currents from about 5 $\mu$A to somewhere well above 200 $\mu$A.
Since it is non-linear at the lower currents, we have introduced a set
of amplifiers with differing gains (x3 and x10) allowing the
non-linear region to be extended to lower currents at the expense of
saturation at the very high currents. Hence there are 3 signals coming
from each BCM (Upx1, Upx3, Upx10, Dnx1, Dnx3, Dnx10). All 6 signals
are fed to scaler inputs of each spectrometer (E-arm and H-arm)
. Hence we have a redundancy of 12 scaler outputs for determining the
charge during a run. During calibration runs we calibrate each of
these scaler outputs.
}%\infolevone

\begin{safetyen}{10}{10}

In coordination with the Hall C run coordinator, all Hall C members
are authorized to take BCM calibration data using the Standard
Non-Invasive Hall C BCM Calibration Procedure which is maintained in
the accelerator document database and it is executed by operates. The
extended calibration procedures involving the Faraday Cup 2 and the
OLO2 monitor at the Injector are presently performed by accelerator
operations though a knowledgeable physicist needs to be present either
in the operations room and/or the counting house to ensure that the
data is recorded and verified as being correct.

\obsolete{
The Accelerator EES group performs the maintenance of the BCM monitors. These 
include:

\begin{tabular}{l l}
1. The Unser calibration. & Every 3 months \\
2. Resonant Cavities Tuning. & Every Downtime \\
3. Multi-meters Autocalibration. & Every Downtime \\
4. Connectors Cleaning. &  Every year \\
5. Unser Keithley Current Source. & Calibration Yearly \\
6. Digital Multi-meters HP3458A and HP 34401A. & Calibration Yearly\\   
\end{tabular}
}

\subsection{Hazards and Mitigations}

As operators perform the calibration procedures there is no hazard to
Hall C personnel in performing a beam current measurement.

\subsection{Responsible Personnel}
System responsible personnel are shown in Table~\ref{tab:BCM:personnel}.
\begin{namestab}{tab:BCM:personnel}{BCM: authorized personnel}{%
   Beam Current Monitor: authorized personnel}
  \DaveMack{\em Contact}
  \JohnMusson{Accel. expert}
\end{namestab}
\end{safetyen}
%Jean-Claude Denard -x 7555

%%% The following is from the old Hall C manual
% There are more BCMs than in the past.  Should these be described too?

\subsection{Current Measuring Devices}

The current in the incoming electron beam must be measured in order to
normalize the counts measured in the spectrometers. There are essentially
three devices that measure currents for Hall~C.

\subsubsection{Unser Monitor}  These are also called Paramagnetic Current Transformers,
or PCT's. This device is the best absolute current transducer available to us.
It is however necessary to calibrate the zero offset of the device at
regular intervals using a wire and a precision current source. This device
is installed at the end of the Hall~C arc.
\subsubsection{RF-Cavity Monitor} This is a tuned RF cavity and the beam current
is measured by measuring the power of the RF radiation coupled in the cavity.
These provide very good relative measurements of current. There are three
of these
devices in the Hall~C beamline. BCM1 and BCM2 are at the location of the Unser
monitor and BCM3 is located on the last beamline girder
immediately preceding the target.  (BCM3 is known as the ``Hall~C" cavity 
by the MCC.)

%\subsubsection{Ion Chamber} This device will provide an absolute measurement of
%the current at low currents. It has not yet been installed but will reside
%in front of the Hall~C beamdump.

%All questions concerning beam current measurements should be
%directed to Dave Mack - x7442.
