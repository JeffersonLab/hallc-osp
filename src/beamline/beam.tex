% Operations Manual Chapter Title 
\infolevone{
\section{Introduction}
\label{sec:beam-intro}
}
% ESAD Section Title
\infoleveqnull{\section{Beamline}}

The control and measurement equipment along the Hall C beamline consists of 
various elements necessary to transport beam with the required specifications 
onto the reaction target and the dump and to simultaneously measure the 
properties of the beam relevant to the successful implementation of the 
physics program in Hall C.  

\infolevone{
The resolution and accuracy requirements in Hall 
C are such that special attention is paid to the following:
\begin{list}{\arabic{enumi}.~}{\usecounter{enumi}\setlength{\itemsep}{-0.15cm}}
  \item Determination of the incident beam energy;
  \item Control of the beam position, direction, emittance and stability;
  \item Determination of the beam current;
  \item Determination of the beam polarization.
\end{list}

Drawings of the Hall C line from the shield wall to the target are
shown in Figs.~\ref{fig:Cline1},~\ref{fig:Cline2},~ \ref{fig:Cline3}, and~ref{fig:Cline4}. 

\begin{figure}
\begin{center}
\includegraphics[angle=0,width=15cm]{compton_12gev.png}
\caption[Beamline: Hall C Beamline Overview]{Compton Polarimeter, located
in the alcove.}
\label{fig:Cline1}
\end{center}
\end{figure}

\begin{figure}
\begin{center}
\includegraphics[angle=0,width=15cm]{moller_12gev.png}
\caption[Beamline: Hall C Beamline Overview]{Schematic of the Hall C
Moller Polarimeter, consisting of a superconducting magnet that polarizes
the Moller target, 3 quadrupole magnets that function as a spectrometer to
analyze Moller scattered electrons, and detectors.  The 3 quadrupoles also
serve as beam optics elements during normal operations.}
\label{fig:Cline2}
\end{center}
\end{figure}

\begin{figure}
\begin{center}
\includegraphics[angle=0,width=15cm]{hallc_beamline_arc_alcove.pdf}
\caption[Beamline: Hall C Beamline Overview]{Hall C beamline from green shield wall to hall entrance.}
\label{fig:Cline3}
\end{center}
\end{figure}

\begin{figure}
\begin{center}
\includegraphics[angle=0,width=15cm]{ds_beamline.png}
\caption[Beamline: Hall C Beamline Overview]{View of the beamline from
entrance to the hall to the target scattering chamber.}
\label{fig:Cline4}
\end{center}
\end{figure}
}

\infolevtwo{
\subsection{The Beam Entrance Channel}

The beam entrance channel consists of 63.5 mm inner diameter stainless steel 
tubing connected with conflat flanges. Through magnets the inner diameter of the 
tubing is restricted to 25.4 mm. 
Each section has a roughing port and is pumped with an ion pump. 
The pressure is about 10$^{-6}$ Torr or better.
}

\subsection{The Beam Optics Channel}

These consist of dipoles, quadrupoles, sextupoles (generally not used), and 
beam correctors with their 
standard girders and stands. Starting from the beam switch yard, there are 
eight dipoles in the arc section which (along with five  other smaller beam 
deflectors) bend the beam 37.5 degrees into the hall. Each dipole has a 
quadrupole and a pair of steering magnets (correctors) associated with it. 
After the shield wall at the entrance to the tunnel into the hall the beam is 
essentially undeflected onto the target and into the dump. However a small vertical
displacement of the beam ($\approx$ 2 cm up) is implemented to
adjust the beam height to the optical axis of the Hall C spectrometers.

The beamline optics elements are designed to deliver 
various optical tunes of the beam on to the physics target as well as 
simultaneously deliver various optical tunes at other locations along the 
beamline. During normal operations, it is possible to deliver beam
with a focus at the physics target in the hall and at the interaction
point of the Compton polarimeter. It is not possible to simultaneously
achieve a focused beam at the M\o ller polarimeter target.

During normal operations, the beam is delivered with an achromatic tune.
For measurement of the beam energy, a dispersive tune is used. The nominal
Hall C beam properties (achromatic tune) are listed in Table \ref{beam_tab3}. 
 
\begin{table}[hp]
\begin{center}
\begin{tabular}{|c|c|c|c|} \hline
{\bf Emittance}  & {\bf Energy Spread} & {\bf spot size}  & {\bf Halo}  \\
   (nm-rad)       &   $\sigma$ (\%)     & $\sigma$ ($\mu$m)&  (\%) \\ \hline 
$\epsilon_x <$ 10 &  $<$ 0.05           & $\sigma_x<$ 400  &  $<$0.01 \\
$\epsilon_y <$ 5  &  $<$ 0.03           & $\sigma_y<$ 200  &        \\ \hline
\end{tabular}
\end{center}
\caption[Beamline: Hall C Beam nominal properties at target]{Hall C Beam nominal properties at target}
\label{beam_tab3}
\end{table}

\subsection{Beam Diagnostic Elements}

The key beam diagnostic elements consist of beam position
monitors (BPMs), beam current monitors (BCMs), and wire 
scanners (harps). Hall C uses harps similar to those used throughout
the CEBAF accelerator - there are several harps placed in the Hall C
beamline which provide both absolute position and beam size information.
The harps at 3C07 and 3C17 are used as part of the beam energy measurement
procedure.  The harp at 3C20 provides information relevant for the Compton
and M\o ller polarimeters. Finally the harps on the ``Hall C'' girder just before
the target (3CH07A and 3H07B) provide position information for calibration of the
BPMs and beam size information. All the harps in the Hall C beamline are controlled
by accelerator. When harp scans are required for the experiment in the hall, shift
workers should contact MCC, request a scan of the relevant harp, and request that
they post the result in the electronic logbook.

To determine the position and the direction of the beam on the experimental 
target point, three Beam Position Monitors (BPMs) are located at distances
3.71 m (IPM3H07A), 2.25 m (IPM3H07B) and 1.23 m (IPM3H07C) upstream of the target position. 
The BPMs consist of a 4-wire antenna array of open ended thin wire striplines 
tuned to the fundamental RF frequency of 1.497 GHz of the beam ~\cite{bi:bar90}. The 
standard difference-over-sum technique is then used ~\cite{bi:HW} to determine the 
relative position of the beam to within 100 microns for currents
above 1 $\mu $A. The absolute  position of the BPMs can be calibrated with respect to the 
superharps which are located adjacent to each of the BPMs (IHA3H07A 
at 3.46 m and IHA1H07B at 1.55 m upstream of the target).

The BPMs are typically read out in two ways.

\infolevone{
1. The averaged position over 0.3 seconds is logged into the EPICS~\cite{EPICSwww} database (1 
Hz updating frequency) and injected into the datastream every few seconds, 
unsynchronized but with an reference timestamp. From these values we can 
consider that we know the average position of the beam calculated in the EPICS 
coordinate system which is left handed.

2. Event-by-event information from the BPMs are recorded in the CODA datastream
from each of the 8 BPM antennas (2x4) from which the position of the beam can be 
reconstructed. However, these raw values belong to a parallel electronics chain 
whose constants have to be retrieved by calibrations to the EPICS or scanner 
data.

Figure~\ref{fig:bpm_readout} shows a schematic of how the BPMs are read out for use
in the Hall C data acquisition.

\begin{figure}
\begin{center}
\includegraphics[angle=0,width=15cm]{bpm_electronics_chain.pdf}
\caption{Readout chain for the beam position monitors in Hall C.}
\label{fig:bpm_readout}
\end{center}
\end{figure}


\subsection{Beam Exit Channel}
% add figure for ds beamline?

After the target vacuum chamber, there is an exit beam pipe which 
transfers the scattered beam onto the dump tunnel under vacuum. The exit beam pipe
will have more than one configuration due to the need to accommodate the use of the smallest
SHMS angle in some cases as well as the need to incorporate magnetic shielding to
prevent the beam from being deflected due to stray fields from the SHMS.

The downstream beam pipe has an overall length of about 90 feet from the exit of the
scattering chamber to the dump entrance. The portion of the pipe closest to the dump
entrance is made of several sections of 24-inch diameter aluminum tube. Some sections
of this 24-inch pipe can be replaced with sections to be used as beam position monitors
(BPMs). Closer to the target, the beam pipe steps down to an 18-inch diameter tube. The combined
length of the 18 and 24 inch diameter sections in about 44 feet.

Moving upstream the next section of pipe has a 6 inch diameter and is about 23 feet long. The
16-feet long section between the scattering chamber exit and the 6-inch diameter pipe is the region
that undergo the most potential configuration changes. Depending on the minimum angles
required for the SHMS and HMS, the beam pipe diameter can vary from 1.5 to 4 inches. As noted
earlier, special magnetic shielding will be required when the SHMS is at small angles (typically
less than 10 degrees), depending on the momentum.

An overview of the downstream Hall C beamline is shown in Fig.~\ref{fig:ds_beamline}.

\begin{figure}
\begin{center}
\includegraphics[angle=0,width=15cm]{hallc_ds_beamline_eng.pdf}
\caption{Schematic of the beam pipe that runs between the Hall C scattering chamber and the
beam dump entrance.}
\label{fig:ds_beamline}
\end{center}
\end{figure}

}

\infolevone{
\subsection{Machine/Beamline protection system}
\label{sec:beam-fsd}

The MPS~\cite{MPScebaf} system is composed of the Fast Shutdown System (FSD), Beam Loss 
Monitor (BLM), and gun control system.

The FSD system is a network of permissive signals which terminate at the 
electron gun and chopper 1. The permissive to the gun and chopper
1 may be inhibited by any device connected to an FSD mode. Devices connected to the 
FSD system include vacuum valves, RF systems, Beam loss systems, beam current 
monitors, beam dumps, and particular to Hall C, the target motion mechanism 
and the raster.

The gun control system includes software program which monitors beam 
operating conditions and the state of the FSD and BLM systems. the program 
will warn the operators if a potential for beam damage exists. Potential for 
damage exists when running high average current beam, when FSD nodes are 
masked and when the beam power approaches the operating envelope limits for a 
specific beam dump.
}

\begin{safetyen}{0}{0}
\infolevone{\subsection{Safety Information}}
%
% Information for the ESAD
%

The beamline in the Hall provides the interface between the CEBAF accelerator
and the experimental hall.   All work on the beamline must be coordinated 
with both physics division and accelerator division; in order to ensure
safe and reliable transport of the electron beam to the dump.

\infolevone{\subsubsection{Hazards}}
\infoleveqnull{\subsection{Hazards}}

Along the beamline various hazards can be found.  These include
radiation areas, vacuum windows, electrical hazards, magnetic fields
and conventional hazards.

\infolevone{\subsubsection{Mitigations}}
\infoleveqnull{\subsection{Mitigations}}

All magnets (dipoles, quadrupoles, sextupoles, beam correctors) and beam 
diagnostic devices (BPMs, scanners, Beam Loss Monitor, viewers) necessary for 
the transport of the beam are controlled by Machine Control Center (MCC) 
through EPICS~\cite{EPICSwww}, except for special elements which are addressed in the 
subsequent sections. The detailed safety operational procedures for the Hall 
C beamline should be essentially the same as the one for the CEBAF machine 
and beamline.\\ 
  
\noindent{}Personnel who need to work near or around the beamline should keep in mind the potential hazards:
\begin{itemize}
  \item Radiation ``Hot Spots'' - marked by ARM or RadCon personnel,
  \item Vacuum in the beam line tubes and other vessels,
  \item Thin windowed vacuum enclosures (e.g. the scattering chamber),
  \item Electric power hazards in vicinity of the magnets,
  \item Magnetic field hazards in vicinity of the magnets, and
  \item Conventional hazards (fall hazard, crane hazard etc.).
\end{itemize}

These hazards are noted by signs and the most hazardous 
areas along the beamline
are roped off to restrict access when operational.   
In particular, the scattering chamber, with it's large
volume and thin windows requires hearing protection once it has been evacuated.   
Signs are posted by RadCon for any hot spots along the beamline and
RadCon must be notified before work is done in a posted area.

Where appropriate (such as for the M\o ller polarimeter magnets),
magnet leads are covered with plastic guards for electrical safety. 

\noindent{}Additional safety information is available in the following documents:
\begin{list}{--}{\setlength{\itemsep}{-0.15cm}}
  \item EH\&S Manual~\cite{EHScebaf};
  \item PSS Description Document~\cite{PSScebaf}
  \item Accelerator Operations Directive~\cite{AODcebaf};
\end{list}

\infolevone{\subsubsection{Responsible Personnel}}
\infoleveqnull{\subsection{Responsible Personnel}}

Since the beamline requires both accelerator and physics personnel to maintain
and operate and it is very important that both groups stay in contact that any 
work on the Hall C beamline is coordinated.

\begin{namestab}{tab:beam:personnel_beam}{Beam line: authorized personnel}{%
   Beamline physics division and accelerator division points-of-contact.}
  \namestabheader{Hall C Physicists}
  \DaveGaskell{\em 1st Contact}
  \MarkJones{\em 2nd Contact}
  \namestabheader{Liaisons from Accelerator Division}
  \HariAreti{..to Physics}
  \JayBenesch{..to Hall-C}
\end{namestab}
\end{safetyen}

