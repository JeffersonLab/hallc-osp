\chapter{Experiment Specific Equipment}
\label{chap:expspecific}



\section{Overview}
For Experiments E12-06-110 and E12-06-121, the standard Hall C cryo
target is replaced by a polarized ${}^3$He target.

The general hazards and hazards associated with the Hall C base

equipment are discussed in chapters~Target}

\subsection{Hazards}
Hazards include:
\begin{itemize}
\item Personnel eye sight damage due to exposure to ifrared laser
  light
\item Fire due to operation of high power lasers
\item Fire due to operation of target oven
\item Explosion of the high pressure target cell or reference cell
\item Activation of target by the electron beam

\subsection{Mitigations}
\begin{itemize}
\item All areas with possible exposure to laser light are enclosed and
  interlocked.  Work with exposed laser light is limited to trained
  personnel wearing appropriate PPE.  The Hall CANS system limits
  access to trained personnel during such work.
\item Temp sensors

\item Target area is surrounded by shields to protect against flying
  debries in the event of a target or reference cell explosion.  When
work in the target area is required,
lasers locked off and hearing protection and face protection required
on the target platform


\subsection{Responsible Personnel}

Points of contact for the Polarized ${}^3$He Target are listed in Tab.~\ref{tab:personnel_threehe}.

\begin{namestab}{tab:personnel_threehe}{Polarized ${}^3$He Target: authorized personnel}
   {Polarized ${}^3$He Target points-of-contact.}
  \JianPingChen{\em 1st Contact}
  \ArunTadepalli{}
  \JunhaoChen{\em Student}
  \MingyuChen{\em Student}
  \MelanieRehfuss{\em Student}
  \MurchhanaRoy{\em Student}
\end{namestab}



References:


LSOP - Laser Operational Safety Procedure
Laser Safety Procedure for the Polarized 3He Target in Hall C and Laser Room

OSP - Operational Safety Procedure
Operatino Safety procedure fo the Polarized 3he Target in Hall C and Laser Room

