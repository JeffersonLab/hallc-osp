% Uncomment and change to title of equipement (e.g. NPS)
\chapter{Neutral Particle Spectrometer}
\section{Overview}
Experiments using the Neutral Particle Spectrometer (NPS) use the Hall C base
equipment with several modifications:
\begin{enumerate}
\item A calorimeter with over 1000 $\textrm{PbWO}_4$ crystals is mounted
  on the SHMS carriage or an extension platform attached to the
  carriage.
\item A normal resistive magnet, with fields up to XX kG, is placed in
  between the target and the calorimeter to sweep charged particles
  out of the path to the calorimeter.
\item For some NPS configurations, an extension platform is installed
  on the small angle side of the SHMS carriage.  This extension
  includes a rail system allowing the calorimeter to be placed at
  various distances from the target.
\item The Horizontal Bender, the first magnet on the SHMS, is removed
  in order to accommodate the calorimeter and sweeping magnet.
\end{enumerate}

Many of the hazards and hazard mitigations of the NPS equipment are
the same or similar to those for the Hall C base equipment.  These
hazards are listed below with references to other sections of this
ESAD where appropriate.

\textbf{Add descriptions to each section below, noting any unique hazards.}
\section{SHMS Carriage and Extension}
Note here that during running, there may not be railings on the
platform in the area of the detector.  And thus work on the detector
may be limited to designated personnel with appropriate training
working under the term of an OSP covering NPS platform.
See section \ref{sec:carriage}.
\section{Sweeper Magnet}
See section \ref{sec:strongmagneticfields}
\section{Calorimeter High Voltage}
See section \ref{sec:esadhighvoltage}


