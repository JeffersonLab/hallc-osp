\chapter{Experiment Specific Equipment}
\label{chap:expspecific}



\section{Overview}
For Experiments E12-06-110 and E12-06-121, the standard Hall C cryo
target is replaced by a polarized ${}^3$He target.

The general hazards and hazards associated with the Hall C base
equipment are discussed in Chapters~\ref{chap:generalhazards}
and~\ref{chap:hallspecificequipment} of this ESAD and in the Jefferson
Lab Hall C Standard Equipment Manual~\cite{HallCosp}.

The target for these experiments consists of a target glass cell containing a
high density of ${}^3$He gas
($\approx 2.5 \times 10^{20}~\textrm{nuclei/cm}^3$)
and a small amount of Rb-K vapor, a reference cell and a solid target
ladder.  The target and reference cells are presurized to $\approx$10 atm.
The target includes an oven used in the polarization of the
gas and is surrounded by Helmholtz coils that define the direction of
polarization.  High power laser light is used to optically pump the rubidium
and potassium atoms which transfer their polarization to the ${}^3$He
via spin exchange.
The target is also surrounded by an enclosure that prevents the escape of
laser light and protects against explosion of the target cell.

Multiple lasers are located in the laser room, located next to the counting
house.  The laser light is transported to the target area over
fibers.  Outside of the laser room, all paths for the laser light are
inside conduit or enclosures.

The target and lasers, their hazards and mitigations, and operation
are fully described in Operational Safety Procedure ``Operation Safety procedure fo the Polarized
3he Target in Hall C and Laser Room'' (Target OSP) and the Laser
Operational Safety Procedure ``Laser Safety Procedure for the
Polarized 3He Target in Hall C and Laser Room'' (LOSP).  Work on these
systems is limited to authorized personnel (Table~\ref{tab:personnel_threehe}).

\subsection{Hazards}
Hazards include:
\begin{itemize}
\item Personnel eye sight damage due to exposure to infrared laser
  light
\item Fire due to operation of high power lasers
\item Fire due to operation of target oven
\item Explosion of the high pressure target cell or reference cell
\item Activation of target by the electron beam
\end{itemize}

\subsection{Mitigations}
\begin{itemize}
\item All areas with possible exposure to laser light are enclosed and
  interlocked.  Work with exposed laser light is limited to trained
  personnel wearing appropriate PPE.  The Hall CANS system limits
  access to trained personnel during such work.
\item Temperature sensors on laser fiber junctions and sensors in the
  laser optics housing are interlocked with the laser shutdown
  system.
\item   The Hall VESDA (Smoke detection) system is also connected
  to the laser shutdown system.  The standard hall cameras and a
  dedicated target camera allow visual monitoring of the target
  apparatus from the counting house.
\item Target area is surrounded by an enclosure to protect against flying
  debries in the event of a target or reference cell explosion.  When
work in the target area is required and gas cells are present, the
target area will be barricaded and limited to trained personnel with
appropriate PPE.
\item Radworker II training is required for personnel working on the
  target after a cell explosion.
\end{itemize}

\subsection{Responsible Personnel}

Points of contact for the Polarized ${}^3$He Target are listed in Tab.~\ref{tab:personnel_threehe}.

\begin{namestab}{tab:personnel_threehe}{Polarized ${}^3$He Target: authorized personnel}
   {Polarized ${}^3$He Target points-of-contact.}
  \JianPingChen{\em 1st Contact}
  \ArunTadepalli{}
  \JunhaoChen{\em Student}
  \MingyuChen{\em Student}
  \MelanieRehfuss{\em Student}
  \MurchhanaRoy{\em Student}
\end{namestab}

Additional contact information may be found on the counting house
whiteboard.  Most staff members can be reached by text message by
visiting the JLab Staff Directory,
\url{https://misportal.jlab.org/mis/staff/}, and employging
the ``Click to page'' option.

