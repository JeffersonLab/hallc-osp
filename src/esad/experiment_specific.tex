
\chapter{Experiment Specific Equipment}
\label{chap:expspecific}


\section{Overview}
For Experiment E12-11-107, the HMS and SHMS spectrometers will detect electrons in
coincidence with the dedicated Large Acceptance Detector (LAD) which detects protons
and neutrons. LAD will be calibrated and monitored using a laser system. Two GEM
tracking planes will be located on the target platform to provide additional tracking
information on the recoil particle. The standard Hall C scattering chamber is rotated
$110^\circ$ compared to its nominal setup so that a larger, 15-in. GEP window can be used on
the SHMS side to act as a window for the LAD detectors.

\section{SHMS Spectrometer Rotation}
For the first half of the experimental run (E12-11-107), the LAD GEM detectors on the target platform limit the minimum angle of the SHMS rotation.  Therefore all SHMS spectrometer rotations will be executed by magnet experts and technicians during a scheduled Hall access.  Remote rotation from the counting house will be disbled.  HMS rotation will be enabled remotely with the normal limitations as defined at rotation checkout.

For the second half of the run, (R-SIDIS E12-06-104 \& E12-24-001), the GEM detectors will be removed while the LAD scintillators will remain in place.  Spectrometer rotation will be enabled remotely for the SHMS and HMS, with the normal limitations as defined at rotation checkout.



\section{GEM Detectors}
Two GEM chambers will be used on the target platform next to the scattering
chamber with sensitive areas of 1.2 × 0.55 m$^2$. The plane closest to the target will be
located at 0.75 m from the target, and the second plane will be located at 0.95 m from
the target. The GEM trackers will cover $67^\circ$ of the in-plane region and $34^\circ$ of the out-
of-plane region. The readout equipment APV25s are located below the target platform,
and the processing electronics are housed in the SHMS hut.
\subsection{Hazards} 
Hazards associated with the operation of GEM detectors to both equipment and personnel are described:
\begin{itemize}
\item Exceeding the recommended high voltage values applied to the detectors.
\item Exceeding the recommended voltage values applied to readout electronics.
\item Exceeding the gas flow rates/input pressures to the detectors leading to the rupture
of the detectors.
\end{itemize}
\subsection{Mitigations}

Any work on this system must be covered by ePAS Permit to Work(s) (PTW).
\begin{itemize}
\item Hazards to personnel are mitigated by turning off HV and LV before doing any
  work on the detectors.
\item Hazards associated with high voltage, both to personnel and detectors, is also
mitigated by the use of HV voltage modules with a maximum current limit of 1
mA. Furthermore, a more strict pre-set current limit of 0.8 mA is programmed into
all channels. Any current exceeding this limit causes the that HV channel to trip
off and remain off.
\item Hazards to the detectors associated with excessive gas flow rates and input pressures
  will be mitigated by the use of pressure relief valves and alarms that will shut off HV if flow/pressure is inadequete
\end{itemize}
  \section{Responsible Personnel}
\label{sec:personnelGEM}
Individuals responsible for the system are:

\begin{table}[!htb]
 \centering
 \begin{tabular}{|c|c|c|c|c|}
\hline
 Name&Dept.&Phone&email&Comments \\ \hline
 Florian Hauenstein & Jlab & & \href{mailto:hauenst@jlab.org}{\nolinkurl{hauenst@jlab.org}} &  First/Laser contact\\ \hline
Carlos Ayerbe Gayoso  & ODU & & \href{mailto:gayoso@jlab.org}{\nolinkurl{gayoso@jlab.org}} & Second contact \\ \hline
 L.  Weinstein & ODU &  &\href{mailto:weinstein@jlab.org}{\nolinkurl{weinstein@jlab.org}}& Contact  \\ \hline
 Holly Szumila-Vance & FIU &  &\href{mailto:hszumila@jlab.org}{\nolinkurl{hszumila@jlab.org}}& \\ \hline
 \end{tabular}
\caption{Personnel responsible for GEM  Detector.} 
\label{tb:gem}
\end{table}


  

\section{Large Angle Detector}

The Large Angle Detector (LAD) is placed in Hall C to the beam left between $5-6$ m away of the target covering angles from $90^{\circ}$ to $160^{\circ}$. 
 
LAD is constructed from refurbished CLAS TOF scintillators used during the 6-GeV era. The scintillators are arranged in three sections with the backward two sections consisting of double scintillator planes (five planes in total).  Each plane consists of 11 bars with a thickness of $5.08\,\mathrm{cm}$, a width of $22\,\mathrm{cm}$ and a length from 4 to 5 m.
All bars are read-out on both sides by PMTs (Philips XP4312B) giving a total of 110 active channels. 

In order to operate the PMTs,  high voltages are provided by a multi-channel CAEN mainframe up to negative 2500V.
Each PMT has two outputs,  a dynode and amplified output.  One output is sent to flash-ADCs (250 VXS,  16 channels/board,  made and owned by JLab) while the other one is sent to discriminators (16 channels/board,  designed by JLab).
The discriminated time signal goes to a TDC (CAEN VX1190A, 128 channels/board, 100 ps/channel resolution).  The read-out system is installed within a bunker at around $90^{\circ}$ angle on the beam left side on the hall floor.

In total,  the DAQ system consists of one VXS crate with 7 flash-ADCs,  7 discriminators,  one TDC and a trigger logic board (CAEN v1495).  Furthermore,  a signal distribution card for the flash-ADCs and trigger interface boards are installed in the crate.  A trigger for cosmics and for a signal from the laser calibration system is implemented one the trigger logic board and particularly feed to the main trigger system of Hall C in the counting house.

Each bar is also connected to a laser calibration system.
The laser calibration system consists of a Photonics STV-01E-140 picosecond pulse laser with a
wavelength of $355\,\mathrm{nm}$,  several splitters,  reference photodiode,  and a fiber distribution
system to each bar.  The main components of the system are in a sealed,  light-tight box within the LAD DAQ bunker.  The typical output of the laser is about $1\,\mu\mathrm{J}$ per pulse at $0.3\,$ns width (FWHM).  The output is then attenuated and distributed via fiber to a 100-channel splitter which is installed in the center of the LAD frame.  After the splitter each fiber has an output power of about $200\,\mathrm{pJ}$.  The fibers are connected via a patch panel on to each scintillator bar. The control of the system is via a remote-controlable raspberry PI from the counting house.

\indent
\subsubsection{Electrical Hazard}

The electrical hazard to personnel can come from the high voltage that powers the PMTs,  which need
up to 2500 V to function. 

\subsubsection{Fall Hazard}

Fall hazard from maintenance and testing operations of LAD that require the use of ladders or scissor lift to access
system elements in up to 6\,m. 

\subsubsection{Laser Hazard}

The laser hazard comes from the laser calibration system and its connection with fibers to the scintillator
bars when work is done on LAD.  The system itself is closed,  light-tight and the output on each fiber is
$\approx 200\,\mathrm{pJ}$.  This intensity is comparable to that of a LED,  however,  the $355\,\mathrm{nm}$
wavelength could be damaging to the human eye since it is invisible to the eye and the natural eye reflex
will not be triggered. 

\subsection{Mitigations}

Any work on this system must be covered by ePAS Permit to Work(s) (PTW).

\indent
\subsubsection{Electrical Hazard Mitigations} 

The electrical hazard associated with the HV system is mitigated by the use of properly rated cables
that are terminated at the voltage dividers and the HV supplies.
The HV supplies are grounded to their electronics racks as well. 
The maximum current provided by the HV distribution boards is quite low ($<1\,\mathrm{mA}$).  The HV system
is designed to shutdown any channels that show an over-current condition. The HV boards must not be accessed
during operation; during maintenance work,  performed by trained personnel,  the HV is turned off,  and the
power supply is switched off by the power switch on the back of the crate.

\subsubsection{Fall Hazard Mitigations} 

Fall hazard mitigation consists of approriate training like fall protection training and harness training.  For individuals using
ladders,  they are required to take the appropriate ladder training.

\subsubsection{Laser Hazard Mitigations}

The laser calibration system is a closed system with fibers connected to the scintillator bars via a patch
panel.  All of these connections are light-tight.  Furthermore,  no fiber must be disconnected from the system
while it is operating.  While the laser intensity in each fiber is very small ($\approx 200\,\mathrm{pJ}$),
and comparable to that of a LED, the $355\,\mathrm{nm}$ wavelength could be damaging to the human eye.
Therefore,  one should not look directly at the fiber output - eye protection must be worn. 

Warning labels are applied on the enclosure of the laser system,  the patch panels,  and detector frame.  The
box containing the laser system is interlocked such that if the box is open the power is off.  If the system
is not in use by trained personnel,  it will be powered off.  During maintenance work on the laser system,
performed by trained personnel,  the laser system is turned off,  disconnected from the power supply,  and
if necessary LOTO is applied.  The procedures for maintenance work on the laser system can be found in the LOSP and specific ePAS Permit to Work(s).

\subsection{Responsible Personnel}
\indent

Individuals responsible for the LAD system are (see Table~\ref{tb:lad}):


\section{Responsible Personnel}
\label{sec:personnel}
Individuals responsible for the system are:

\begin{table}[!htb]
 \centering
 \begin{tabular}{|c|c|c|c|c|}
\hline
 Name&Dept.&Phone&email&Comments \\ \hline
 Florian Hauenstein & Jlab & & \href{mailto:hauenst@jlab.org}{\nolinkurl{hauenst@jlab.org}} &  First/Laser contact\\ \hline
Carlos Ayerbe Gayoso  & ODU & & \href{mailto:gayoso@jlab.org}{\nolinkurl{gayoso@jlab.org}} & Second contact \\ \hline
 L.  Weinstein & ODU &  &\href{mailto:weinstein@jlab.org}{\nolinkurl{weinstein@jlab.org}}& Contact  \\ \hline
 Holly Szumila-Vance & FIU &  &\href{mailto:hszumila@jlab.org}{\nolinkurl{hszumila@jlab.org}}& \\ \hline
 \end{tabular}
\caption{Personnel responsible for the Large Angle Detector.} 
\label{tb:lad}
\end{table}

