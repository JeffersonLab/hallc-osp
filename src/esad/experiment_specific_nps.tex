%
% Add ESAD material for experiment specific equipment to this file
%
\chapter{Experiment Specific Equipment}
\label{chap:expspecific}



\section{Overview}
For Experiments E12-13-0010,E12-13-007 and E12-22-006, the Neutral Particle Spectrometer (NPS) will be used in addition with a sweeping magnet.


The general hazards and hazards associated with the Hall C base
equipment are discussed in Chapters~\ref{chap:generalhazards}
and~\ref{chap:hallspecificequipment} of this ESAD and in the Jefferson
Lab Hall C Standard Equipment Manual~\cite{HallCosp}.

The NPS calorimeter is a calorimeter made of 1080 PbWO4 arranged in an array of 36x30. The calorimeter  
 sits on an additionnal platform located on the beamline side of the SHMS.
The calorimeter, magnet, their hazards and mitigations, and operation
are fully described in Operational Safety Procedure ``Operation Safety procedure fo the Polarized
NPS Calorimeter Operation'' (NPS calorimeter OSP)

%Work on these
%systems is limited to authorized personnel (Table~\ref{tab:personnel_threehe}).

\subsection{Hazards}
Hazards include:
\begin{itemize}
\item Beamlime damage when rotating SHMS to change NPS calorimeter angle
\item Electrical hazard when operating the detector
\item Fire due to shorts in electronics
\item Magnetic field when sweeping magnet is turned on
\item Water leaks
\end{itemize}

\subsection{Mitigations}
\begin{itemize}
\item SHMS rotation limit switches will be adjusted to avoid any interference with beamline or HMS
\item HV part are enclosed in the calorimeter box which is interlocked with HV supply. No parts at HV are accessible when box is closed.
\item  The low voltage and high voltage power supplies have trip levels preventing major shorts.  The Hall VESDA (Smoke detection) will alarm in case o fbeginning of fire.  The standard hall cameras would allow people on shift to see any occurence of smoke from the detector apparatus from the counting house.
\item a check list will be carried out before turning on magnet to check for any metallic pieces nearby the magnet. A ``Magnet On'' display will turn on when field is on
\item the chiller system is using water for controlling the temperature of the calorimeter blocks. Water leaks will be visually checked with the camera and a water detector will be placed on the floor of the calorimeter.
\end{itemize}

Any work on this system must be covered by an ePAS Permit To Work (PTW)

\subsection{Responsible Personnel}

Points of contact for the NPS calorimeter are listed in Tab.~\ref{tab:personnel_threehe}.

\begin{namestab}{tab:personnel_threehe}{NPS experts}
   {Polarized ${}^3$He Target points-of-contact.}
  \SimonaMalace{\em 1st Contact}
  \CarlosMunoz{}
  \WassimHafidi{\em Student}
\end{namestab}

Additional contact information may be found on the counting house
whiteboard.  Most staff members can be reached by text message by
visiting the JLab Staff Directory,
\url{https://misportal.jlab.org/mis/staff/}, and employging
the ``Click to page'' option.

