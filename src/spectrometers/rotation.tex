\section{Spectrometer Carriage and Rotation Systems}

The Carriage is the support structure of the spectrometer.

Each entire spectrometer can be rotated. Rotation is driven by motors mounted
near one of the sets of wheels. These motors are controlled by synchronous
pulse width modulated drives which are mounted near the bottom of the shield
house steps on the HMS, and under the rear of the SHMS structure.

\infolevone{
The spectrometer angles are found using a reference plumb bob and/or TV camera
attached at a known location under the rear of each spectrometer.
This camera is focused on survey marks scribed into plates which are attached to
the rail on the floor. Using the scribe plates and a vernier scale, the angles of the
spectrometers may be determined with a resolution of approximately $0.01^{\circ}$.
}

Since the Hall C spectrometers each weigh hundreds of
tons, it is very important that all safety precautions are carefully
adhered to.   During operations, the spectrometers are certified
to allow remote rotation by shift crews within prescribed limits.  In
the absence of this certification, the spectrometers may only be
rotated by trained technical staff.


\begin{safetyen}{0}{0}
\subsection{Hazards}

Hazards include:
\begin{itemize}
\item{Knocking items over during spectrometer movement.}
\item{The wheels crushing things (including fingers and toes) on the floor in the path of the
spectrometer}
\item{Damaging the beamline or other equipment on the floor if one goes to too small
or too large an angle.  There is
only a small gap between the rear of the SHMS shield house and the shielding wall
behind it.}
\item{Tearing out of cables etc. physically attached to the superstructure}
\item{Elevated platforms on the spectrometer carriages.}
\item{Magnetic fields.  Spectrometer magenets may be energized under either
local or remote control.}
\end{itemize}

\subsection{Mitigations}
Hazard mitigations:
\begin{itemize}
	\item{Stop-blocks attached to the rails to prevent spectrometer rotation beyond
	the needed angular range for each experiment.}
	\item{During experiments, the spectrometers are certified for rotation by
	shift crews within specied angle ranges.  Spectrometer movement at other times
	may only be performed by authorized personnel.}
	\item{Hand rails are installed to prevent falls.  Access is only allowed to
	areas on the carriage protected by hand rails.}
	\item{Hard hats may be required under certain conditions when working on or
	near the spectrometer carriages.}
	\item{Magnetic fields hazards are indicated by either red flashing lights and/or
	illuminated \textbf{Magnet on} signs.}
\end{itemize}
\end{safetyen}
Any work on this system must be covered by an ePAS Permit To Work (PTW)

\infolevone{
\subsection{Remote Rotation}

Prior to the start of experiment operations, Hall-C staff will certify that the spectrometers
are configured for safe remote rotation by verifying all required clearances and
implementing interlocks, mechanical stops, or administrative controls, as appropriate.
Refer to the section of this manual on \emph{Controls} to find instructions for
rotating the spectrometers. The cameras enable remote readout of the HMS/SHMS
spectrometer angles using the survey marks on the floor.
The wide-lens and zoom cameras located at the entrance and
the exit of Hall~C should be used to visually search for obstructions
before and during remote rotation.

Limit switches are installed at forward
and backward angles which prevent HMS (SHMS) from rotating to angles more forward than
10.6 (5.5) degrees and wider than 85 (40) degrees. To obtain more forward or wider
angles an access is needed, and rotation has to occur manually
downstairs with spotters. Hard limit switches will be installed to prevent
the spectrometer from rotating out of maximum allowed range.

In case the spectrometers are rotated to more forward angles, pay special
attention to possible interferences of HMS and SHMS, and interference between either
spectrometer and the beam pipes or their stands.

%Remote rotation  is accomplished via the PLC located in the Hall.  It
%is currently installed in the HMS shield hut for protection from
%radiation.  Commands can be issued to the PLC (Texas Instruments 5000), which executes these commands
%following algorithms stored in its memory. An EPICS screen is now
%used to talk to the PLC.  The advantages of using the PLC are:
%
%\begin{itemize}
%\item{No direct access by users to the algorithms, preventing unsafe
%rotation attempts (instead, the algorithms have to be loaded locally into
%the PLC).}
%\item{Rotation of both HMS and SOS by the same smart controller, enabling
%security checks of both angle decoders. This renders a better handle on the
%minimum allowed angle between the two spectrometers.}
%\item{Automatic slower rotation speeds (if desired) if close to desired angle
%when using proximity switches.}
%\end{itemize}
%
%The PLC communicates directly with the control electronics of several limit
%switches, proximity switches, and decoders. Next to the limit switches
%also hard limit switches will be installed on the floor to mitigate failure
%of the PLC limit switches.
}%\infolevone

%%%%%%%%%%%%%%%%%%%%%%%%%%%%%%%%%%%%%%%%%%%%%%%%%%%%%%%

\infolevone{\subsubsection{Responsible Personnel}}
\infoleveqnull{\subsection{Responsible Personnel}}

Following the experimental run plan, as posted in the counting house
by the run coordinator, shift workers are allowed to rotate the
Hall C Spectrometers
following guidelines of the standard equipment manual.  In the event
of a problem getting the spectrometers to rotate the run coordinator
should notified.  If the run coordinator is unable to solve the
problem, and with the run coordinators concurrence, qualified
personnel should be notified to repair the problem (see
Table~\ref{tab:spectrometers:personnel_rotate}).

\begin{namestab}{tab:spectrometers:personnel_rotate}{Spectrometer Rotation: authorized personnel}{%
      List of Spectrometer Rotation responsible personnel where ``W.B.'' stands for the white board
      in the counting house.}
   \TechonCall{\em Contact}
   \SteveLassiter{}
   \JerryNines{}
   \JoeBeaufait{}
\end{namestab}

\infolevone{
\subsection{Personnel Trained for Manual Spectrometer Rotation}

The spectrometer motors may only be manually controlled by trained personnel. if it
is necessary to rotate the spectrometer manually, contact one of the trained
personnel listed in Table~\ref{tab:spectrometers:personnel_rotate}.
At least two people are required for manual spectrometer rotation: one to
run the motors and at least one spotter. Prior to rotating the spectrometer
a visual inspection of the area must be made to insure that there
is nothing in the spectrometer's path or on the rails. During rotation
the spotter should pay special attention to
the cables which run from the
spectrometer to the target motor controller to make sure that
nothing is hung up or stretching.
}
