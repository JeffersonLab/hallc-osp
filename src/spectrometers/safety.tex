\begin{safetyen}{0}{0}
\infolevone{\section{Safety Information}}
\label{sec:spectrometer-safety}

\subsection{Hazards}

The spectrometers have associated vacuum, electrical, cryogenic and
magnet systems all of which can be extremely dangerous due to the size
and stored energy in the systems.
Parts of the spectrometers are at elevated levels which would present fall
hazards if the installed safety equipment were not present.
Hazards of rotating the
spectrometers as well as the particle detectors that get placed inside
the detector hut of the spectrometer are covered in detail in
following sections.

Signage and alerts are placed to remind workers of some of the potential hazards
in Hall C, but each individual is ultimately responsible for his or her own
safety. Always read and respect warning signs, and never attempt to
circumvent barriers or other equipment
that has been installed for your protection. If you discover what appears to be a
new or unidentified hazard, protect your coworkers by warning them
and alert the Hall-C management and Safety Warden.

\subsection{Mitigations}

Both of the spectrometers have elevated work platforms that are secured by
gates and handrails. Never attempt to bypass these protections. During experiment
running periods, in order to allow spectrometer rotation, it may be necessary to
remove the handrails around the target platform. In this condition access to the
target platform is restricted to trained individuals who have been specifically
authorized to work near there. Fall-protection equipment is required.

The vacuum systems associated with the spectrometers are essentially
pressure vessels and care should be exercised so as not to damage or puncture the
vacuum windows.   The large vacuum windows inside the two shield houses are protected
by shutters which must be lowered into place before the access door to the
detector rooms will open. (When the Noble-Gas Cherenkov (NGC) is installed in the SHMS,
the NGC itself protects the vacuum window and the shutter is not present.)
During hall maintenance, covers are placed over the spectrometer
vacuum windows near the pivot to help prevent anything from accidentally hitting
a window. Hearing protection may be required when you work near a vulnerable
vacuum window. As conditions may change, please take note of currently posted
warning signs and instructions.

The magnets themselves are installed inside cryostats.  These vessels
are exposed to high pressures and are therefore equipped with safety
relief valves and burst discs.

The cryogenic system operates at an elevated pressure and at temperatures
about 4~Kelvin (helium system) and about 90~Kelvin (nitrogen system).  One must
guard against cold burns and take the normal precautions with pressure
vessels when operating or working near this system.  Manipulation of any cryogenic
system component such as a U-Tube or manual valve may only be performed by
a trained cryogenic-system expert.

When they are powered, the magnets have a great deal of stored energy as they are large
inductors.
\infolevone{(See Table~\ref{tab:magnet_parameters}.)}
Always make sure people are clear of
the magnets and their dump resistors.

\subsection{Responsible Personnel}

In the event that problems arise during
operation of the spectrometers, qualified personnel should be notified
(see Table \ref{tab:spec:personnel_technical}).
This includes any prolonged or serious problem with the source of magnet
cryogens (the ESR).
\infolevone{(See also ``Checking Cryogenics'' in Section \ref{ssec:operatemagnets}, below.)}
On weekends and after hours there will be a
designated individual on call for magnet services.  Any member of the
Hall C technical staff is qualified to deal with unusual magnet
situations but in the event of serious problems the person on
call should be contacted.

\begin{namestab}{tab:spec:personnel_technical}{Spectrometers: authorized personnel}{%
      List of spectrometer responsible personnel where ``W.B.'' stands for the white board
      in the counting house.}
   \TechonCall{\em Contact}
   \SteveLassiter{}
%   \EricSun{}
%   \MikeFowler{}
   \JoeBeaufait{}
%   \JackSegal{}
   \HeidiFansler{}
%   \MahlonLong{}
\end{namestab}


\end{safetyen}
