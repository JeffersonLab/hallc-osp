\section{Hall~C Spectrometer Magnets}

The basic parameters describing the magnets of the HMS and SHMS are provided in
Table~\ref{tab:magnet_parameters}.

The quadrupoles determine the focusing properties of the spectrometers
and to a large extent their acceptance.
To achieve the lowest possible scattering-angle
settings for each spectrometer, both Q1's are
made asymmetric: narrow horizontally and elongated in the vertical direction. For the same
reason a notch is present in the outer mantles of both Q1 vacuum cans so
that the incident electron beam passes through this notch when the
spectrometer is at its smallest angle ($12.5^{\circ}$ for HMS and $5.5^{\circ}$ on the SHMS).

In order to reach such a small scattering angle, the SHMS is equipped with the HB
(horizontal bend) dipole magnet. It bends the central-ray tracks $3^{\circ}$ away from the
beamline so that the remainder of the spectrometer components are aligned $8.5^{\circ}$
from the beam when the spectrometer is rotated to center on a $5.5^{\circ}$ scattering angle.
When used at scattering angles below about $12^{\circ}$, the stray magnetic fields from
the SHMS HB, Q1, and Q2 magnets can deflect the primary electron beam, possibly
causing it to miss the beam dump, unless the schemes designed to prevent this are followed.

The main dipole is the dispersive element in each spectrometer system. It
determines the central momentum of the spectrometer.
%The present operations envelope states that the HMS dipole power supply may not be
%operated at currents above 1300 Amps. This corresponds to a central
%momentum of about 4~GeV/c.



\begin{table}
  \begin{center}
    \caption{Characteristics of the SHMS Magnets\label{tab:magnet_parameters}}
    %\vspace{\baselineskip}
    \resizebox{\textwidth}{!}{%
      \begin{tabular}{|l|c|c|c|c|c|r|r|r|r|}
        \hline
	&		&		&			&\multicolumn{4}{|c|}{Values at Maximum Momentum}		\\
	&     		& Eff. Len.	& Aperture	&\multicolumn{1}{|c|}{Mom.}	&\multicolumn{1}{|c|}{Current}	&\multicolumn{1}{|c|}{Field or}	&\multicolumn{1}{|c|}{Energy}\\
	&     		&(m)		& (cm)		&\multicolumn{1}{|c|}{(GeV/c).}	&\multicolumn{1}{|c|}{(A)}		&\multicolumn{1}{|c|}{Gradient}	&\multicolumn{1}{|c|}{(MJ)}\\
        \hline
        HMS Q1	& Cold Iron	& 1.89	& $\phi~40$	&7.4		& 980	&	& 0.23	\\
        HMS Q2	& Cold Iron	& 2.10	& $\phi~60$	&7.4		& 830	&	& 0.81 \\
        HMS Q3	& Cold Iron	& 2.10	& $\phi~60$	&7.4		& 370	&	& 0.16	\\
        HMS D	& Wam Iron	& 5.26	& $42 (w)$	&7.4		&3000	&2.03~T	& 9.9	\\ \hline

        SHMS HB	& ``C" Septum	& 0.752	& $14.5 \times 18$	&11	& 3930	&2.56~T	& 0.2		\\
        SHMS Q1	& Cold Fe		& 1.86	& $\phi~40$	&11		&2455	&7.9~T/m	& 0.39	\\
        SHMS Q2	& $cos(2\theta )$& 1.64	& $\phi~60$	&11		&3630	&11.8~T/m& 7.6	\\
        SHMS Q3	& $cos(2\theta )$& 1.64	& $\phi~60$	&11		&2480	&7.9~T/m	& 3.4	\\
        SHMS D	& $cos(\theta )$& 2.85	& $\phi~60$	&11		&3270	&3.9~T	& 13.7	\\ \hline
      \end{tabular}
    }
  \end{center}
\end{table}

\subsection {Magnet Cryogenics}

The HMS and SHMS magnets are all
superconducting and hence their coils must be maintained at
cryogenic temperatures during operations. The LHe required by the magnets
is supplied by the End Station Refrigerator, ESR.
All of the spectrometer cryogenic services are supplied through the overhead
cryogenic lines. The distribution network begins at the distribution
box over the pivot. This box is connected to the HMS and SHMS networks via the
flexible transfer lines over the pivot. The network is adjacent to
the upstairs catwalk on the HMS and, on the SHMS, is along the small-angle side
of the upper platform.

Cryogenic information about each magnet is available on the control
screens located in the Hall~C counting house.

All of the magnets were originally designed to be cryostable, meaning that they
cannot quench unless the level of liquid helium drops below the coils. Testing of
the HMS magnets up to 4~GeV/c settings has shown that this design goal
was achieved. Nevertheless, the energy stored in the field of each magnet is sufficient
to cause an unrecoverable quench if all of it were dumped into the magnet.
Therefore, every superconducting spectrometer magnet in Hall~C, even though
cryostable, is protected by a quench protection circuit. This circuit safely dissipates
much of the magnetic-field energy in a high-power dump resistor. The SHMS
magnets have not received extensive cryostability testing yet.

\subsection {Magnet Power Supplies}

The power supplies for the magnets are located on the carriage
adjacent to the magnets. The supplies are all water cooled and
the water flow rate to the supplies can be seen on the water flow
meter located near the electronics boxes on the floor near the pivot.
This meter views the flow for all the HMS power supplies
(dipole and quads) and a reading of 33 $\%$ corresponds to approximately
20 gallons per minute through the combination of supplies (they are supplied
in parallel).
As of writing, the total flow rate for the SHMS power supplies has not been
measured. When it is measured, the nominal value will be available in the Hall-C electronic log book
and the manual you are reading will be updated.

The front panels of the power supplies are interlocked. Under
no circumstances should the front panel of any supply be opened by anyone other
than authorized personnel.

When the supplies are energized there are flashing red lights  and
illuminated "Magnet On" signs placed at
several locations on the HMS and SHMS carriages to alert personnel to the magnet
status. There are also signs posted listing the dangers of high magnetic
fields.

The control interfaces for the power supplies are available in the
Hall~C counting house on the HMS and SHMS control screens. Experimenters
use these screens to control the magnet power supplies. The magnet experts
will sometimes use controls that are local to the power supplies.

\subsection {Magnet Personnel}
In the event that any problems arise during operations of the magnets
of either the HMS or the SHMS,
one of the qualified responsible personnel listed in
Table~\ref{tab:spec:personnel_technical} should be consulted.  This includes any prolonged
or serious problem with the source of magnet cryogens (the ESR).
On the weekend and after hours there
will be a designated individual on call for magnet services. This person should be
contacted first.


%\subsection{Quadrupole Magnets}
%
%The quadrupoles determine the transverse focusing properties of the spectrometers
%and to a large extent their acceptance.
%
%All three quadrupoles for the HMS spectrometer are cold iron superconducting
%magnets. The soft iron around the superconducting coil enhances the field at
%the coil center and reduces stray fields.
%The basic parameters for the first quadrupole, Q1, are an effective (actual)
%length of 1.89 (2.34) meter and an inner pole radius of
%25.0 centimeter. \cite{bi:yan1}
%The vacuum vessel
%inner radius for Q1 is 20.05 cm. To achieve the lowest possible angle
%setting of the HMS spectrometer (with respect to the beam line), Q1 is
%made asymmetrical, and is elongated in the vertical direction. For the same
%reason a notch in the outer mantle of the Q1 cryo vessel is made, such
%that the incident electron beam passes through this notch when the
%HMS spectrometer is at its smallest angle of 12.5 degrees.
%The other two quadrupoles, Q2 and Q3, are essentially identical with an
%effective (actual) length of about 2.10 (2.60) meter and an inner pole radius
%of 35.0 centimeter. For these quadrupoles the vacuum vessel inner radius
%amounts to 30.0 centimeter.
%
%% These will need to be updated for whatever max momentum we certify
%The maximum operating currents (assuming a 4 GeV/c momentum particle) for the
%quadrupoles are about 580 A, 440 A, and 220 A, for Q1, Q2, and Q3, respectively.
%To establish a correct focusing onto the detector plane with
%the quadrupole triplet we may want to cycle the quadrupoles to about 20\% higher
%current values, rendering maximum currents of 700 A, 530 A, and 270 A,
%respectively. This will render pole field values of 1.25, 1.30, and 0.65 T,
%respectively.
%The energy stored in the quadrupole fields is sufficient to cause an
%unrecoverable quench if all the energy stored is dumped into the
%magnets. \cite{bi:hms1}
%Therefore a quench protection circuit is incorporated. However, a quench
%can only happen if the cryomagnets have a helium level below the coil during
%operation.
%
%The operating current to the quadrupole coils is provided by three
%Danfysik System 8000 power supplies, which can operate up to 1250 A current
%and 5 V voltage. The power supplies are cooled with a combined maximum
%water flow of 45 liters per minute.
%
%In addition to the main quadrupole windings, all quadrupoles have multipole
%windings to make corrections to the fields.  These mulitpole windings
%are no longer used and are not connected to power supplies.
%%To further optimize focusing properties of the HMS magnet system
%%we may operate some of these multipole trim coils
%%\cite{bi:yan2}.
%%The operating current for these multipole
%%corrections is small (the multipole corrections are typically less than
%%2\% of the main quadrupole field), of order 50 A, and are provided by
%%three HP power supplies. These power supplies can operate up to 100 A current
%%and 5 V voltage.
%
%%%
%
%\newpage
%\subsection{Dipole Magnet }
%
%The dipole is the dispersive element in the system and
%determines the central momentum of the spectrometer.
%The present operations envelope states that the supply may not be
%operated at currents above 1300 Amps. This corresponds to a central
%momentum of $\approx$4 GeV/c.
%
%The dipole for the HMS spectrometer is a superconducting, cryostable magnet.
%Its basic parameters are an effective length of 5.26 meter,
%a bend radius of 12.06 meter, and a gap width of 42 cm.
%Its actual size is 5.99 meter long, 2.75 meter wide, and 4.46 meter high.
%It is configured to achieve a 25 degree bending angle for 4 GeV/c momentum
%particles at a central field excitation of 1.11 T.
%For the HMS dipole to reach 1.11 T the maximum operating current for the coil
%amounts to 1300 A.
%
%The dipole has been designed to achieve cryostability up to a field of 2 T,
%and this property has been extensively tested up to a field of 1.11 T.
%The cryostable coils are equipped with an energy removal circuit to cover
%the possibility of an unrecoverable quench. \cite{bi:hms2}
%However, this can only happen
%if the helium level drops below the coil during operation.
%The current to the coils will be provided by a Danfysik System 8000 power
%supply, which can operate up to 3000 A current and 10 V voltage.
%This power supply is located on the carriage beside the dipole, and
%is cooled with a maximum water flow of 35 liters per minute.
%The flow of the magnet cooling water is regulated by flow meters installed
%on the floor of Hall~C. The total water flow needed to cool the 4 power
%supplies for the HMS magnet system (dipole and quadrupoles) amounts
%to 80 liters per minute, with a supply pressure of cooling water for
%Hall~C of 250 psi.


\subsection{Operation of the Spectrometer Magnets}
\label{ssec:operatemagnets}

HMS and SHMS magnet controls have been extensively revised. See
section~\ref{sec:spectrometercontrols}
for instructions. The controls and
monitoring screens are accessed through a GUI/HMI that magnet system
experts will have initialized in the Counting Room.

The information immediately following is being temporarily retained and is likely
covered in more detail elsewhere.

\subsubsection{Setting Magnet Currents}

The polarities of the currents in the HMS and SHMS magnets are such that
HB, Q2 and DIPOLE have the same sign as the charge of the particles
to be transmitted. Q1 and Q3 have the other sign. If you use the ``standard
tune" setting on the controls GUI this will be handled for you automatically.

While in the past users of Hall~C had to run a program called ``field" to obtain
the predicted current or field settings for a given spectrometer momentum,
these parameters are now determined for you by the controls program. You
need only to enter the desired momentum in the GUI.

One thing has not changed:
\begin{itemize}
\item{Wait at least 7 minutes for the HMS dipole magnet to settle.}
\end{itemize}

Up to this moment we have not witnessed any clear signature of hysteresis
effects for the HMS dipole magnet. For the HMS quadrupole magnets a small effect
on the field has been witnessed, but only for low currents (typically smaller
than 100 A). A procedure for setting the quadrupoles was developed
and shown to achieve a high
degree of reproducibility in setting the quads at low current.
\\
\\
\textbf{The HMS Quadrupole Cycling Procedure:}
\begin{enumerate}
\item{On every change of polarity, take the magnet up to 950 Amps
  (in the new polarity!), then down to zero before setting the
  current.}
\item{To set the current the first time after a polarity change
  go up to 200 Amps higher than the desired current,
  then down to the desired current.

  This means: to change the polarity and set the current go to 950 Amps,
  down to zero, back up to 950 Amps, and down to the desired setting.}
\item{Subsequently:
  \begin{itemize}
  \item{Changes to lower currents can be made directly.
    That is, just set the magnet for the lower current.}
  \item{For changes to higher current, first overshoot
    by 200 Amps, then come back down to the desired current.}
\end{itemize}}
\end{enumerate}
This procedure is called \textbf{CYCLING THE MAGNET}, and needs to be  followed for all three HMS quadrupoles.

Commissioning of the SHMS magnets with beam has not occurred as this is written, although it
is expected that only SHMS Q1 might need to by cycled. The other SHMS magnets have
current-dominated fields (rather than iron-dominated) and are not expected to
exhibit significant hysteresis effects. Cycling procedures, if needed, will be established by
the Physics Division Liaison in consultation with the engineering group and collaborators.

\subsubsection{Checking Cryogenics}

\begin{description}
\item{\bf 1}\hskip0.1in Routine checking
\item{}\hskip0.3in The HMS and SHMS magnets all operate with liquid-level controlled
  reservoirs. It is therefore sufficient to verify that the liquid level
  is near the set point to be assured of cryogenic happiness.
\end{description}

\begin{description}
\item{}\hskip0.3in The setpoints are all 70\%.  The liquid
  level is normally within a few percent of this value.  If
  the helium level is significantly above this
  the helium reservoirs are overfilling.  This is not harmful and the
  levels will return to normal in several hours.  If the level is
  significantly below the set points (5\% or more) there is usually
  something wrong.  Selecting a time graph of liquid level is helpful in
  determining if the situation is a temporary fluctuation or if the
  situation is serious.
\item{\bf 2}\hskip0.1in Helium Problem Resolution
\item{}\hskip0.3in If helium liquid level is observed falling, check the cryogen status of all nine magnet
  systems.  If multiple systems are losing liquid helium,  call CHL x7405 as the
  likely cause is a site wide problem.  CHL will advise if recovery is
  short (1-2 hours) or much longer.  If the recovery is short do nothing!
  If the recovery is long then it can be beneficial to make some
  adjustments in Hall~C.  This requires an access and a knowledgeable
  individual: the on-call Hall~C magnet responsible person
  should be summoned. Refer to Table~\ref{tab:spec:personnel_cryo}.
\item{\bf 3}\hskip0.1in Single System Failures
\item{\bf 3.1}\hskip0.1in Single System Loss of LN$_2$
\item{}\hskip0.3in If a single system is observed losing LN$_2$ you can
  wait until
  the next day to call someone in as the LN$_2$ usage of all the magnets
  is extremely low.  They can go for 24 hours without a refill. Remember that the
  HB magnet on the SHMS uses no LN2, so do not let that confuse you.
\item{\bf 3.2}\hskip0.1in Single System Loss of LHE Level
\item{}\hskip0.3in This is usually caused by a single computer failure or
  components failure.  Call the on-call magnet responsible person (Table \ref{tab:spec:personnel_cryo})
  and plan an access to Hall~C. The dipole reservoir will go empty in 1 hour so a quick reaction is
  necessary.  The quads take much longer, 4 hours or more to empty
  allowing more time to react.  All of the magnets have low level
  interlocks that will automatically safely discharge the magnets
  so that you can safely operate until they are ``dry."
\item{\bf 4}\hskip0.1in Temporary Loss of LN$_2$ To All Systems
\item{}\hskip0.3in Occasionally during site LN$_2$ delivery, the supply to
  Hall~C
  is temporarily stopped.  This can be checked by calling ESR Cryo Coordinator
  (see Table \ref{tab:spec:personnel_cryo}).  There
  can be local Hall~C problems that result in loss of LN$_2$ to the
  magnets. The ``call in" can be deferred to a convenient time for this
  kind of problem.
\end{description}

\begin{namestab}{tab:spec:personnel_cryo}{Spectrometers: magnet cryogenic experts.}{%
    In case magnet cryogenic issues, contact the Hall C
    engineer on call and the cryogenic group if necessary.
    To contact the cryogenic group during working hours, try
    David Schleeper, Joe Wilson, or the guard shack.  Outside of working hours, contact
    the guard shack at x5822 and ask for a call back from the on call cryo engineer.}
  \EngonCall{}
  \SteveLassiter{Cryo Expert}
  \EricSun{Cryo Expert}
%  \MikeFowler{Cryo Expert}
%  \AndyKenyon{Cryo Expert}
  \CryoonCall{}
  \DavidSchleeper{ESR Cryo Coordinator}
  \JoeWilson{Cryo Operations}
\end{namestab}


%
%\begin{table}
%\begin{center}
%\caption{Current Liquid Level Settings\label{tab:liq_levels}}
%\vspace{\baselineskip}
%\begin{tabular}{|l|l|l|}
%\hline
%{} & {}  & {}  \\
%{} & LHE & LN2 \\
%{} & {}  & {}  \\ \hline
%Q1  & 75\% & 75\% \\
%Q2  & 75\% & 75\% \\
%Q3  & 75\% & 75\% \\
%Dipole& 70\% & 70\%/65\% (Hi/Low) \\
%\hline
%\end{tabular}
%\end{center}
%\end{table}
%
