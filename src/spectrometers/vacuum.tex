
\section{Spectrometer Vacuum Systems}

	Each spectrometer contains several separate vacuum volumes.
These are the isolation vacuums of each of the superconducting
magnets and the main spectrometer volumes through which the particles
to be detected travel.

The isolation vacuums are normally cryo-pumped by the cold mass when the
magnets are cold and hence have no mechanical pumps
associated with their maintenance. They also have no thin
windows or other hazards and will not be discussed further in this document.

The main spectrometer vacuum of the HMS
is maintained by the two mechanical pumps.
One pump is located between Q3 and the Dipole on the small angle
side of the carriage while the second pump is the backing
pump of the turbo. It pumps the spectrometer volume through the turbo.
This mechanical pump is located
at the back of the carriage and the turbo is located in the shield house
underneath the end of the vacuum can that protrudes into the detector hut.
The vacuum in the HMS channel may be read out on a gauge that is sitting on the
carriage beneath Q3. There is a TV camera that views this gauge readout and
it is displayed on one of the TV monitors  in the Hall~C counting house.

A similar installation creates the vacuum in the SHMS channel. The mechanical
pump sits on the beamline-side of the SHMS carriage below the
upstream end of the dipole magnet. It is connected to the vacuum channel
through a port on the bottom of the plenum on the downstream end of the
dipole, inside the shield house. The vacuum level is monitored by the
readout included on the SHMS controls GUI.

If either spectrometer vacuum starts to deteriorate rapidly an expert should be notified.
Vacuum system responsible personnel are listed in Table ~\ref{tab:spectrometers:personnel_vacuum}.



