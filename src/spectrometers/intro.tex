%\section{Introduction}

	The two magnetic spectrometers in Hall~C are designed to perform
high resolution and high accuracy nuclear physics experiments.  These
spectrometers transport and detect charged particles that are scattered or produced
in beam-target interactions.
\ifdefined\SHMSMAGNETSOSP
In this document we discuss the features and safe operation of
the Super High Momentum Spectrometer, or SHMS.
After this introduction and overview of the SHMS, the
SHMS (and the legacy HMS - High Momentum Spectrometer)
are discussed together as a series of subsystems.
This document
\else
In this chapter
we discuss the features and safe operation of these spectrometers,
the High Momentum Spectrometer, or HMS,
and Super High Momentum Spectrometer, or SHMS.
These spectrometers
share many common features.  After this
introduction and overviews of the individual spectrometers, the
spectrometers are discussed together as a series of subsystems.
This chapter covers the ``mechanical'' subsystems of the spectrometers.
This chapter
\fi
covers the ``mechanical'' subsystems of the spectrometers.
The detector packages and
shield houses are discussed in
\ifdefined\SHMSMAGNETSOSP
Chapter 5 of the Hall C Standard Equipment Manual~\cite{HallCosp}.
\else
Chapter~\ref{chap:detectors}.
\fi

\ifdefined\SHMSMAGNETSOSP
The SHMS is
\else
Both the HMS and the SHMS are
\fi
designed around a series of superconducting magnets,
including quadrupoles and dipoles, followed by a set of particle detectors.  The first
magnet on the SHMS is a horizontal-bending (HB) dipole that bends particles
away from the beam-line.
\ifdefined\SHMSMAGNETSOSP\else
The HMS does not have a horizontal bender.
\fi
The primary purpose of the quadrupole magnets is to increase the flux of
charged particles entering the main dipole magnets and to focus the orbits of the
charged particles into the detector huts.
The dipole magnets deflect charged particles vertically
as they enter the detector huts. Some of the detectors measure the amount of deflection
so that we may
determine the momentum of each particle. Other detectors provide accurate timing
to trigger the readout of all detector data, or they measure a particle's speed
or total energy so that we can determine its mass.

\ifdefined\SHMSMAGNETSOSP
The SHMS
\else
Each spectrometer
\fi
is built on a rotatable support structure or ``carriage".
These ride on steel
wheels and rails and rotate around the pivot. Experiments using the HMS or
SHMS place their experimental targets along the beamline above the pivot.
The support structures carry the weight of the magnets and detectors, and keep
them aligned to one-another and pointed at the target. On the SHMS, the shield
house and all of the
other spectrometer equipment are also carried by the support structure.
\ifdefined\SHMSMAGNETSOSP\else
The
HMS has a separate carriage that supports the weight of the shield house.
\fi

During beam operations a many particles are scattered and
produced in the target. A few of these enter one of the spectrometers. Most of
them go forward at small angles and are stopped in the beam dump. The
remainder, scattered over a wide range of angles,  interact with the beam pipe,
the air, nearby structures, etc., and
constitute background radiation that would overwhelm the detectors.
The shield houses have thick walls that are designed to reduce the amount of
radiation that gets inside. Both spectrometers use
concrete and lead lining on the walls as shielding. On
the SHMS, two custom types of concrete improve the neutron stopping power
of the walls by
adding boron (in the form of boron-carbide) or extra hydrogen (recycled
plastic chips). Each shield house has a room that surrounds the detectors.
The SHMS has a second room, shielded from the first, that protects the
electronics of the DAQ system and the magnet control systems.

%Each spectrometer includes a shield house which contain the particle
%detector packages.  In the case of the SHMS, this shield house (and a separate
%shield house for electronics) are part of the carriage structure.  In
%the case of the HMS, the detector shield house is a separate
%structure, coupled to the carriage by a ``push bar'' on the ``pasta fork''.
%The ``pasta fork" is the name given to the steel piece that protrudes
%from the back of the carriage. The detector frame is supported on the
%``pasta fork." This insures that the detectors do not move relative
%to the magnetic elements.
