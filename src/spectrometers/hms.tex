\section{High Momentum Spectrometer (HMS) }

The HMS, which operates on the beam-right side of the beamline,
includes three superconducting quadrupole magnets and one superconducting dipole magnet. The quadrupoles were
manufactured for JLab by {\em OXFORD} while the dipole was built for
JLab by {\em ELIN}.  The quadrupole magnets are referred to as Q1, Q2,
and Q3, where a particle first traverses Q1, then Q2 and Q3, and
finally traverses the dipole magnet. The dipole of the HMS deflects central-momentum
particle trajectories upwards by $25^{\circ}$.


The magnet system is followed by a large concrete detector hut, in which all
detector elements reside. The main fraction of the detector elements have been
built by universities involved in the Hall~C physics program.  These
detectors are described in chapter~\ref{chap:detectors}.

The HMS spectrometer can be moved to scattering angles between
$10.5^{\circ}$ and $90^{\circ}$. This range is usually constrained by
administrative, software and hardware limits depending on what
downstream beam pipe is installed and what obstructions are currently in the hall.  The
maximum momentum accessible to the HMS magnet
system is presently $6~\textrm{GeV}/c$. This limit is expected to increase
once the HMS dipole has been certified for use at higher momenta during
the 12 GeV era.

