
\section{SHMS Fringe Fields and Small Spectrometer Angles}
The horizontal bender (HB) and second quadrupole (Q2) of the SHMS have
significant fringe fields.  At small scattering angles and
sufficiently high fields, these fringe fields may deflect the
electron beam beyond acceptable limits at the dump and/or
trip dump ion chambers.  To minimize beam deflection and beam
interruptions,
magnetic shielding will be installed to minimize the deflection of the beam and
an accelerator test plan will be executed at the start of the run to determine
any restrictions on SHMS angles and fields.

Under certain combinations of angles and magnet settings, determined
from this test plan,
the shift crew will be required to notify MCC before making changes to
the spectrometer angle or magnet fields and follow a prescribed procedure.
This procedure may include requesting tune (pulsed) beam and masking the
dump ion chambers FSD while new SHMS spectrometer settings are established.

Independent of fringe field considerations, when the HMS and SHMS angles are
below certain limits, or the sum of the HMS and SHMS angles is below a certain
limit, angle changes will require spotters in the Hall to watch for
spectrometer/beamline and spectrometer/spectrometer interferences.



