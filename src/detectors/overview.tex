The SHMS and HMS spectrometers each contain a set of detectors that
trigger event readout and perform tracking and particle identification
(PID).  The detector package in each spectrometer is similar with
differences in part due to the different momentum ranges of the
spectrometers.  These detectors are located in shielded detector huts
to minimize background from the hall.  The detector huts include the
following types of detectors, not all of which may be installed,
depending on the needs of a particular experiment:
\begin{itemize}
\item[]\textbf{Drift Chambers} Two drift chambers, each with 6 wire planes,
separated by about a meter.  These detectors provide tracking of
particles exiting the spectrometer vacuum chamber.  The tracks from
these chambers are traced, with software, back to the target to
provide particle momentum, scattering
angles, and target position.  They are usually the first detectors
in the path of the detectors.  (The SHMS Noble Gas Cerenkov will
sometimes be installed in front of the SHMS drift chambers.)
\item[]\textbf{Hodoscopes} A set of 4 hodoscope planes.  A hodoscope plane
consists of a set of long thin elements/bars, each several centimeters
wide, that covers the acceptance of particles passing through the
detector package.  A pair of planes, one with horizontal bars, one
with vertical bars (S1X, S1Y) are located after the drift chambers.  A
second pair, S2X and S2Y, are located some distance past the pair.
The sensitive elements of these planes are scintillator material, with
the excpetion of S2Y in the SHMS which uses quartz.  The light from
particles passing through the bars is detected by photomultiplier
tubes (PMTs) at the ends of each bar.  The hodoscopes, possibly in
coincidence with other detectors, provide the trigger to the data
acquisition system and a time reference for the drift chambers.
They also contribute to PID by measuring the time of
flight between the two pairs of planes and from pulse amplitude.
\item[]\textbf{Gas Cerenkov detectors} Depending on momentum and index of
refraction of the gas used, provide either discrimination between
electrons and pions, or between pions and kaons
\item[]\textbf{Aerogel Cerenkov} Contain an Aerogel radiator and a diffusion box
  with PMTs to collect Cerenkov light.  These detectors aid in the
  detection of kaons by  discrimination between kaons
and protons (or at low momentum discrimination between pions and
protons.)
\item[]\textbf{Shower Counter} A large array of lead glass.  This detector is
  last in each detector stack.  Most of the energy of electrons or
  positrons is
  collected in the shower counter aiding in the discrimination between
  electrons/positrons and other particles.
\end{itemize}

The detector packages are key to successful measurements.  Their
proper operation should therefore be constantly monitored during
shifts. There are normally a number of diagnostic spectra available to
aid in this process.  Typically each collaboration customizes it's own
set of diagnostic spectra.

