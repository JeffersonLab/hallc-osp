
The detector packages of the SHMS and HMS must perform at least two
functions: tracking and
particle identification (PID).
A particle traversing the detector stack
encounters two sets of drift chambers with six planes of sense
wires in each chamber. From the chamber information both positions
and angles in the dispersive and transverse directions can be determined.
The information from these chambers is the principal input
of the tracking algorithms.

\sawnote{Some details have to be changed to encompase differences in SHMS.}
The chambers are followed by two planes of hodoscopes designated S1X and S1Y.
These plastic scintillator arrays provide the timing reference for the
drift chambers, are used in trigger formation, and in combination
with a second hodoscope pair, provide time-of-flight particle identification.
These scintillators can also be used to perform crude tracking.

The next element encountered by a particle is a gas threshold Cerenkov detector.
This is used for particle identification. In the SOS spectrometer, this
gas threshold Cerenkov detector can be replaced by an Aerogel detector,
with a similar function.

The second hodoscope pair, S2X and S2Y, is located directly behind the
gas Cerenkov. Their function is essentially the same as that of S1X and S1Y.
In the SOS spectrometer, an option exists to have this hodoscope pair
be preceded by a third chamber to improve tracking.

The final element in the detector stack is the lead glass shower calorimeter.
This is used for energy determination and PID.
