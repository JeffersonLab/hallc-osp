
\section{The SHMS Detector Package and Shield House }

The Super High Mometum Spectrometer contains two separate shielded
rooms.  The first room, referred to as the electronics hut, contains
electronics associated with the trigger and data acquisition system as
well as magnet power supply controls.  \sawnote{Refer to a figure with
  a map of the contents.}  This room is entered through a concrete
door on the second stairway landing of the spectrometer structure.
The second shielded room, known as the detector hut, is accessed by
passing through the electronics hut.  These rooms are separated a
sliding concrete door.  Both doors must be closed during beam
operations.  The electronics and detector huts have removable ceiling
and wall sections in case detectors or electronics racks need to be
removed or installed.  Walter Kellner, the Hall~C work coordinator,  
must be contacted if the hut walls or ceilings need to be removed.
\sawnote{Or should we say Hall leader needs to be contacted to remove the door.}

\subsection{Operation of SHMS Electronics Hut and Detector Hut Doors}


\subsection{High Voltage Supplies}
\sawnote{High voltage on first and second floor, mix of new and old mainframes.}

\subsection{Drift Chambers}

\paragraph{Gas Flow Operating Procedures}
\sawnote{This may be common with enough the HMS that it can be moved
  into common chapter.  Or put common stuff (gas shed) in common
  section and hut details in spectrometer specific sections.}

\paragraph{Electronics Operating Procedures}
\sawnote{Description of drift chamber electronics and turn on procedure.
Threshold, +/-5V, VME crate(s).}

\paragraph{High Voltage Operating Procedures}
\sawnote{Not using CAEN?  What are used, are they remotely
  controlled.}

\subsection{Hodoscope}
The SHMS hodoscope consists of 4 planes, the first three made of bars
of scintillating plastic, and the fourth of synthetic quartz.
\sawnote{Should this be combined with HMS or be entirely separate.}
\sawnote{Check that Positive/Negative convention is the same.}

\subsubsection{Scintillator Hodoscope Planes}

\subsubsection{Quartz Bar Hodoscope Plane}
\sawnote{Quartz detector is positive HV.  Big difference from HMS.
  Probably have it's own section separate from HMS.}

\subsection{Noble Gas Cerenkov Detector}
\sawnote{This detector is only sometimes installed}

\subsection{Heavy Gas Cerenkov Detector}
\sawnote{This detector is only sometimes installed?}

\subsection{Aerogel Detector}
\sawnote{Need a table of different index of refraction trays}.
The SHMS Aerogel detector is located after the heavy gas cerenkov
detector and before the S2 hodoscope planes.

Characteristics.  Removable Aerogel trays.  Procedure for exchanging
trays.  Provides Kaon, Proton separation. Used in conjunction with the
other detectors serves to identify kaons.

\begin{table}
\caption{Momentum threshold above which pions, kaons and protons will 
produce Cerenkov light in Aerogel's with various indices of refraction.
\label{tab:shms_aerogel}}
\begin{center}
\begin{tabular}{cccc}
  n& $\pi_{\textrm{thr}}(\textrm{GeV}/c)$ & $K_{\textrm{
  thr}}(\textrm{GeV}/c)$ & $p_{\textrm{thr}}(\textrm{GeV}/c)$\\
  1.030 & 0.57 & 2.00 & 3.80 \\
  1.020 & 0.67 & 2.46 & 4.67 \\
  1.015 & 0.81 & 2.84 & 5.40 \\
  1.011 & 0.94 & 3.32 & 6.31 \\
\end{tabular}
\end{center}
\end{table}



\subsection{Lead Glass Shower Calorimeter}

The SHMS lead glass shower calorimeter consists of two section, a
pre-shower layer of 28 TF-1 lead glass 10 cm thick followed by a fly's eye
array of 224 F-101 blocks with a depth of 50 cm.



