\subsection{Hall C Laser Pulser System}
The Hall C laser pulser sytem provides light pulses
in the visible  blue scintillation region to monitor
the gain stability of the HMS and SOS shower counters and scintillators,
and, at a later stage, the neutron detector of the $G_{En}$ experiment.

\paragraph{UV Laser System}
We use a nitrogen Laser that provides UV laser pulses at 337 nm
from 1 to 50 Hz repetition rate. The output power is 250 $\mu$J.
It is located in a designated area in the Counting House of Hall C.
The laser is enclosed in an aluminum box where the laser light is directed
onto a scintillation fiber after passing through two optical filters
to optimize the light output. 
The laser light is absorbed in the scintillation fiber
and converted to visible blue scintillation light. Both ends of the
scintillation fiber are coupled using CT connectors, mounted on an
aluminum casing, to regular
1 mm diameter multimode plastic optical fibers. One fiber output is
connected to a PIN diode to monitor the light output; the other
fiber is brought downstairs into Hall C where it is connected to 
a distribution box (1:24) located in the cupboard below the pivot. From
this distribution box 4 fibers are running to the HMS spectrometer and
4 to the SOS spectrometer, where they
are connected again to distribution boxes (1:64). The individual
outputs of the final distribution box are connected to the scintillators
and lead glass shower counters. The light pulses in the detector
produce a trigger in the data acquisition system and ADC and TDC
information is read out. Normalizing the peak positions of the light 
pulses to the PIN diode ADC value is a tool to monitor the gain of
each photo multiplier tube throughout the experiments.

\paragraph{Laser Operation} 
In order to turn on the nitrogen laser the right half of the top
plate of the aluminum box needs to be removed. Inside the box 
the laser and the optics is mounted on a single aluminum board.
The lasing part of the laser is confined in the left part of the box
with a separation in the central region of the box. Due to this separation
no UV laser light can escape from this left part of the box,
even during operation, so that the user can safely remove and install the
top plate above the right half of the laser casing.
The laser itself has a security key that needs 
to be in a vertical position when the power switch is pressed. After
2 minutes of warmup-time an orange light turns on indicating that
it is possible now to turn the key into its horizontal position.
After this it takes 15 seconds until the laser actually turns on.
In the event of a power failure the laser will turn off and will not
turn on again after power is restored. It is necessary to turn the key
back to its vertical position and to reset the safety system of the laser.
The laser can now be turned on again by switching the key back to its
horizontal position. The pulse frequency of the laser can be regulated by
turning the black knob at the rear of the laser casing.
