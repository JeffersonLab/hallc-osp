%
% Section for ESAD describing the
% Hall C Gas Cherenkov detectors
%
\infoleveqnull{
\section{Gas Cherenkov Detectors}

% Volumes
% HMS 2.7 m^3
% SHMS HGC 3.4 m^3
% SHMS NGC

The Hall C spectrometer include 3 gas Cherenkov detectors, the HMS
Cherenkov (CER), the SHMS Noble gas Cherenkov (NGC) and the SHMS Heavy Gas
Cherenkov (HGC).  These detectors each contain several cubic meters
of inert gas (CER: 2.7 $\textrm{m}^3$, HGC: 3.4 $\textrm{m}^3$,
NGC: $3.0~\textrm{m}^3$).  The Noble gas detector operates at 1 atm
with a low continuous gas flow for the Hall C gas system.  The HGC and CER
operate at sub-atmospheric pressure.  After filling, the detectors and
their gas supplies are valved off and no gas flows through the detectors.

\subsection{Hazards}
The detectors have phototubes that operate at high voltage with the HGC
operating with negative HV and the CER and NGC with positive high voltage.
The detectors contain inert gases.

\subsection{Mitigations}
Analysis of ODH hazards and gas filling procedures for these three detectors
are discussed in their respective Operational Safety Procedures.

Voltages must be set to zero before disconnecting HV cables from the
detectors.  HV cables must be disconnected before replacing PMTs or bases.

\subsection{Responsible Personnel}

Maintenance of the gas Cherenkov detectors is
systems is routinely performed by the Hall C
technical staff and the university groups that built them.
Shift personnel are not expected to be responsible
for maintaining the detector gas systems (see Table
\ref{tab:gascer:personnel}
for the names of persons to be contacted in case of problems).

\begin{namestab}{tab:gascer:personnel}{Gas Cherenkov detectors: authorized personnel
}{%
      Responsible personnel for gas Cherenkov detectors.}
  \TechonCall{\em Contact}
  \BradSawatzky{}
  \DonalDay{NGC/CER}
  \GarthHuber{HGC}
\end{namestab}
}
